\subsection{Hasonlítsa össze a rendszereket a leírt állapotváltozók (dimenzió) száma (véges/végtelen), valamint annak diszkrét/folytonos jellege szerint!}

\paragraph{A rendszer dimenziója:}
\begin{itemize}
    \item a rendszer állapotváltozóinak száma
    \begin{itemize}
        \item állapotváltozó: az állapot egyértelmű leírására szolgálnak
        \item állapot: a múlt összesített hatása
    \end{itemize}
    \item végtelen: végtelen számú állapotváltozóval írható le a rendszer
    \item véges: véges számú állapotváltozóval írható le a rendszer
    \begin{itemize}
        \item $\underline{x}(t) \in \mathbb{R}^n$
    \end{itemize}
\end{itemize}

\paragraph{Diszkrét/folytonos jelleg:}
\begin{itemize}
    \item diszkrét állapotú:
    \begin{itemize}
        \item ha egy véges dimenziójú rendszer állapotváltozói véges számú értéket vehetnek fel, akkor a rendszer diszkrét állapotú
        \item pl. digitális technika: 0 vagy 1
        \item állapotautomaták:
        \begin{itemize}
            \item az állapotbeli változások egy-egy esemény hatására ugrásszerűen mennek végbe
            \item tipikusan szekvenciális hálózatok
        \end{itemize}
    \end{itemize}
    \item folytonos állapotú:
    \begin{itemize}
        \item az állapotváltozók folytonos értéket vesznek fel
        \item pl. sebesség
    \end{itemize}
\end{itemize}

\begin{figure}[H]
    \centering
    \begin{tikzpicture}
        \begin{scope}[local bounding box=plots]
            % Analog
            \begin{axis}[
                at={(0,4cm)}, anchor=south west,
                width=5cm, height=4cm,
                axis lines=middle,
                xtick=\empty, ytick=\empty,
                xlabel={$t$}, ylabel={$y(t)$},
                title={analóg},
                ymin=0, ymax=1.2,
                xmin=0, xmax=4
            ]
                \addplot[blue, thick, samples=100, domain=0.2:3.8] {0.5 + 0.3*sin(deg(x*2)) + 0.1*sin(deg(x*5))};
            \end{axis}
            
            % Mintavételezett
            \begin{axis}[
                at={(6cm,4cm)}, anchor=south west,
                width=5cm, height=4cm,
                axis lines=middle,
                xtick={1,2,3}, xticklabels={{$k-1$}, {$k$}, {$k+1$}},
                ytick=\empty,
                xlabel={$k$}, ylabel={$y[k]$},
                title={mintavételezett},
                ymin=0, ymax=1.2,
                xmin=0, xmax=4
            ]
                \addplot[blue, thick, ycomb, mark=*, samples at={0.5,1,1.5,2,2.5,3,3.5}] {0.5 + 0.3*sin(deg(x*2)) + 0.1*sin(deg(x*5))};
            \end{axis}
            
            % Kvantált
            \begin{axis}[
                at={(0,0)}, anchor=south west,
                width=5cm, height=4cm,
                axis lines=middle,
                xtick=\empty, ytick=\empty,
                xlabel={$t$}, ylabel={$y_q(t)$},
                title={kvantált},
                ymin=0, ymax=1.2,
                xmin=0, xmax=4
            ]
                \addplot[blue, thick, const plot, samples=20, domain=0.2:3.8] {round((0.5 + 0.3*sin(deg(x*2)) + 0.1*sin(deg(x*5)))*5)/5};
            \end{axis}
            
            % Digitális
            \begin{axis}[
                at={(6cm,0)}, anchor=south west,
                width=5cm, height=4cm,
                axis lines=middle,
                xtick=\empty, ytick=\empty,
                xlabel={$k$}, ylabel={$y_q[k]$},
                title={digitális},
                ymin=0, ymax=1.2,
                xmin=0, xmax=4
            ]
                \addplot[blue, thick, only marks, mark=*, samples at={0.5,1,1.5,2,2.5,3,3.5}] {round((0.5 + 0.3*sin(deg(x*2)) + 0.1*sin(deg(x*5)))*5)/5};
                % dotted lines to show quantization levels?
                \draw[gray, dotted] (0, 0.2) -- (4, 0.2);
                \draw[gray, dotted] (0, 0.4) -- (4, 0.4);
                \draw[gray, dotted] (0, 0.6) -- (4, 0.6);
                \draw[gray, dotted] (0, 0.8) -- (4, 0.8);
                \draw[gray, dotted] (0, 1.0) -- (4, 1.0);

            \end{axis}
        \end{scope}
        
        \node[rotate=90, anchor=south] at ($(plots.north west)!0.25!(plots.south west) + (-0.5,0)$) {térben folytonos};
        \node[rotate=90, anchor=south] at ($(plots.north west)!0.75!(plots.south west) + (-0.5,0)$) {térben diszkrét};
        
        \node[anchor=north] at ($(plots.south west)!0.25!(plots.south east) + (0,-0.5)$) {időben folytonos};
        \node[anchor=north] at ($(plots.south west)!0.75!(plots.south east) + (0,-0.5)$) {időben diszkrét};
    \end{tikzpicture}
\end{figure}

\paragraph{Diszkrét idejű folytonos értékű rendszer:}
\begin{itemize}
    \item differencia egyenletekkel írható le $\underline{x}[k+1] = \underline{\Phi}\underline{x}[k] + \underline{\Gamma}\underline{u}[k]$
    \item az állapotváltozók tetszőleges értéket vehetnek fel, de a diszkrét idő miatt ugrásszerűen mennek végbe
\end{itemize}

\paragraph{Folytonos idejű folytonos értékű rendszer:}
\begin{itemize}
    \item differenciál egyenletekkel írható le $\underline{\dot{x}}(t) = \underline{A}\underline{x}(t) + \underline{B}\underline{u}(t)$
    \item állapotváltozók tetszőleges értéket vehetnek fel, folytonos idő miatt folyamatosan változnak
\end{itemize}