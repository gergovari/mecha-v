\subsection{Mutassa be, hogy az átviteli mátrix hogyan származtatható a lineáris idő invariáns (LTI) rendszerek folytonos idejű állapottér modellből!}

\noindent \textbf{Lineáris idő invariáns rendszerek folytonos idejű állapottér modellje:}
\begin{align*}
    \underline{\dot{x}}(t) &= \underline{\underline{A}}\underline{x}(t) + \underline{\underline{B}}\underline{u}(t) \\
    \underline{y}(t) &= \underline{\underline{C}}\underline{x}(t) + \underline{\underline{D}}\underline{u}(t)
\end{align*}

\begin{itemize}
    \item első lépésben Laplace transzformáljuk az egyenleteket:
    \begin{align*}
        s\underline{X}(s) - \underline{x}(0) &= \underline{\underline{A}}\underline{X}(s) + \underline{\underline{B}}\underline{U}(s) \\
        \underline{Y}(s) &= \underline{\underline{C}}\underline{X}(s) + \underline{\underline{D}}\underline{U}(s)
    \end{align*}

    \item az 1. egyenlet átrendezve:
    $$ (s\underline{\underline{I}} - \underline{\underline{A}})\underline{X}(s) = \underline{x}(0) + \underline{\underline{B}}\underline{U}(s) $$

    \item $\underline{X}(s)$ kifejezése:
    $$ \underline{X}(s) = (s\underline{\underline{I}} - \underline{\underline{A}})^{-1}\underline{x}(0) + (s\underline{\underline{I}} - \underline{\underline{A}})^{-1}\underline{\underline{B}}\underline{U}(s) $$

    \item $\underline{X}(s)$ visszahelyettesítve a kimeneti egyenletbe, majd rendezve:
    $$ \underline{Y}(s) = \underbrace{\underline{\underline{C}}(s\underline{\underline{I}} - \underline{\underline{A}})^{-1}\underline{x}(0)}_{\text{tranziens tag}} + \underbrace{(\underline{\underline{C}}(s\underline{\underline{I}} - \underline{\underline{A}})^{-1}\underline{\underline{B}} + \underline{\underline{D}})}_{\text{átviteli mátrix}}\underline{U}(s) $$

    \item Így az átviteli mátrix ($\underline{U}(s)$ együtthatója)
    $$ W(s) = \underline{\underline{C}}(s\underline{\underline{I}} - \underline{\underline{A}})^{-1}\underline{\underline{B}} + \underline{\underline{D}} $$
    \begin{itemize}
        \item a tranziens tag eltűnik, az állandósult viselkedést az átviteli mátrix írja le
    \end{itemize}
\end{itemize}