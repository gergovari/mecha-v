\subsection{Hasonlítsa össze a vezérlést és a szabályozást: a hatáslánc jellege, zavarjelekkel szembeni ellenállás, a mért mennyiség fajtája, reakció idő, illetve az irányításhoz felhasznált eszközök költsége szerint!}

% Define block styles for diagrams
\tikzset{
    block/.style = {draw, fill=white, rectangle, minimum height=2em, minimum width=3em},
    sum/.style = {draw, fill=white, circle, node distance=1cm},
    input/.style = {coordinate},
    output/.style = {coordinate},
    pinstyle/.style = {help lines,font=\scriptsize}
}

\begin{minipage}[t]{0.48\textwidth}
    \textbf{Vezérlés:}
    \begin{itemize}
        \item nyílt hatáslánc: nincs visszacsatolás
        \item a rendszer belső jellemzőit/bemeneteit mérjük, valamint részben függ a külső feltételektől is, de a kimenetet (irányítani kívánt jellemzőt) nem mérjük
        \item logikai függvényt hozunk létre
        \item gyors reakció
        \item csak determinisztikus zavarok kezelésére képes
        \item vagy egyszerű (olcsó), de így kevés dolgot veszünk figyelembe
        \item vagy komplex (drága) és sok mindent figyelembe veszünk
    \end{itemize}

    \textbf{vezérlés hatáslánca:}

    \begin{center}
        \begin{tikzpicture}[auto, node distance=2cm,>=latex']
            \node [block] (vez) {vezérlő};
            \node [block, right=1cm of vez] (rend) {rendszer};
            \draw [->] (vez) -- node {$u$} (rend);
            \draw [->] (rend) -- node {$y$} ++(1cm,0);
        \end{tikzpicture}
    \end{center}

    \textbf{vezérlés fajtái:}
    \begin{itemize}
        \item idővezérlés – pl. lámpa időzített fel-, lekapcsolása
        \item lefutó vezérlés
              \begin{itemize}
                  \item sorrendi – pl. csomagolás
                  \item feltétel
              \end{itemize}
    \end{itemize}
\end{minipage}
\hfill
\vrule
\hfill
\begin{minipage}[t]{0.48\textwidth}
    \textbf{Szabályozás:}
    \begin{itemize}
        \item zárt hatáslánc: van visszacsatolás
        \item az irányítani kívánt jellemző is tudja befolyásolni a folyamatot
        \item instabil rendszert is képes kezelni (stabilizálni)
        \item lassabb reakció, pontatlanabb
        \item képes kiküszöbölni a zavarjeleket
        \item alacsony műszaki komplexitás:
        \item mérés
        \item kivonás
        \item jelerősítés
    \end{itemize}

    \textbf{szabályozás hatáslánca:}

    \begin{center}
        \begin{tikzpicture}[auto, node distance=2cm,>=latex', scale=0.8, every node/.style={scale=0.8}]
            \node [input, name=input] {};
            \node [sum, right=0.7cm of input] (sum) {};
            \node [block, right=0.7cm of sum] (szab) {szabályozó};
            \node [block, right=1.2cm of szab] (szak) {\begin{tabular}{c} szabályozott \\ szakasz \end{tabular}};
            \node [output, right=1.2cm of szak] (output) {};
            \node [block, below=0.8cm of $(szab.east)!0.5!(szak.west)$] (fb) {\begin{tabular}{c} érzékelő \\ (visszacsatolás) \end{tabular}};

            \draw [->] (input) -- node {alapjel} (sum);
            \draw [->] (sum) -- node {hibajel} (szab);
            \draw [->] (szab) -- node {\scriptsize \begin{tabular}{l} beavatkozó \\ jel \end{tabular}} (szak);
            \draw [->] (szak) -- node [name=y] {\scriptsize \begin{tabular}{l} szabályozni \\ kívánt \\ jellemző \end{tabular}} (output);
            \draw [->] (y) |- (fb);
            \draw [->] (fb) -| node [pos=0.9, left] {mért jel} (sum);
            \draw (sum.south west) -- (sum.north east);
            \draw (sum.north west) -- (sum.south east);
        \end{tikzpicture}
    \end{center}

    \textbf{szabályozás fajtái:}
    \begin{itemize}
        \item értéktartó – pl. hőmérséklet
        \item követő – pl. pálya
        \item kaszkád – több hurkú szabályozás
        \item állapotszabályozás – állapottér modell alapján szabályoz – többváltozós rendszerek – pl. inverz inga egyensúlyban tartása
    \end{itemize}
\end{minipage}