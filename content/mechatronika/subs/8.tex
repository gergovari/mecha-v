\subsection{Adott két diszkrét idejű átviteli függvény. Vezesse le az eredő átviteli függvény összefüggését, ha a két átviteli függvény sorba, párhuzamosan, illetve visszacsatolva (pozitív és negatív) kapcsolódik egymáshoz. Mutassa be, a hatásvázlat átalakításának szabályait (elágazási pont áthelyezése tag mögé és tag elé, illetve összegzési pont áthelyezése tag mögé és tag elé)!}

\noindent \textbf{Hatásvázlat:}
\begin{itemize}
    \item rendszer működését lehet ábrázolni
    \item bal oldalt a bemenet(ek), jobb oldalt kimenet(ek)
    \item grafikus úton kapunk átviteli függvényt
    \item nyíl irányával jelezzük a jel haladási irányát
    \item műveletek:
    \begin{itemize}
        \item elágazás
        \item összegzés (negatív esetén az érintett negyed besatírozása)
    \end{itemize}
    \item az impulzus átviteli függvényeket blokkban ábrázoljuk
\end{itemize}

\noindent \textbf{Sorba kapcsolás:}
\begin{figure}[H]
    \centering
    \begin{tikzpicture}[auto, node distance=2cm, >=latex]
        \node [coordinate] (input) {};
        \node [draw, rectangle, right=1.5cm of input, minimum height=2em, minimum width=3em, thick] (w1) {$\mathbf{w_1(z)}$};
        \node [draw, rectangle, right=1.5cm of w1, minimum height=2em, minimum width=3em, thick] (w2) {$\mathbf{w_2(z)}$};
        \node [coordinate, right=1.5cm of w2] (output) {};

        \draw [->, thick] (input) -- node [above] {$u_1(z)$} (w1);
        \draw [->, thick] (w1) -- (w2);
        \draw [->, thick] (w2) -- node [above] {$y_1(z)$} (output);
    \end{tikzpicture}
\end{figure}
$$ u_1(z) \cdot w_1(z) \cdot w_2(z) = y_1(z) $$

\noindent \textbf{Párhuzamosan kapcsolás:}
\begin{figure}[H]
    \centering
    \begin{tikzpicture}[auto, node distance=2cm, >=latex]
        \node [coordinate] (input) {};
        \node [coordinate, right=1cm of input] (branch) {};
        
        \node [draw, rectangle, above right=0.5cm and 1cm of branch, minimum height=2em, minimum width=3em] (w1) {$w_1(z)$};
        \node [draw, rectangle, below right=0.5cm and 1cm of branch, minimum height=2em, minimum width=3em] (w2) {$w_2(z)$};
        
        \node [draw, circle, right=4cm of branch, inner sep=0pt, minimum size=6mm] (sum) {};
        \draw (sum.north west) -- (sum.south east);
        \draw (sum.north east) -- (sum.south west);
        
        \node [coordinate, right=1cm of sum] (output) {};

        \draw [thick] (input) -- node [above] {$u_1[z]$} (branch);
        \draw [->, thick] (branch) |- (w1);
        \draw [->, thick] (branch) |- (w2);
        \draw [->, thick] (w1) -| node [near end, above] {$y_1[z]$} (sum);
        \draw [->, thick] (w2) -| node [near end, below] {$y_2[z]$} (sum);
        \draw [->, thick] (sum) -- node [above] {$y[z]$} (output);
        
        \fill (branch) circle (2pt);
    \end{tikzpicture}
\end{figure}
$$ y(z) = y_1(z) + y_2(z) = w_1(z)u_1(z) + w_2(z)u_1(z) $$
$$ y(z)(w_1(z) + w_2(z))u_1(z) = w_e(z)u_1(z) $$

\noindent \textbf{Visszacsatolás, elágazási pontok, összegzési pontok:}
\begin{itemize}
    \item negatív visszacsatolást szoktunk leggyakrabban alkalmazni
    \begin{figure}[H]
        \centering
        \includegraphics[width=0.8\textwidth]{res/imgs/mecha_8_neg_feedback_1.png}
    \end{figure}
    
    \item pozitív visszacsatolás:
    \begin{figure}[H]
        \centering
        \begin{tikzpicture}[auto, node distance=1.5cm, >=latex]
            \node [coordinate] (input) {};
            \node [draw, circle, right=1cm of input, minimum size=6mm, inner sep=0pt] (sum) {};
            \draw (sum.north west) -- (sum.south east);
            \draw (sum.north east) -- (sum.south west);
            
            \node [draw, rectangle, right=1cm of sum, minimum height=2em, minimum width=3em, thick] (w1) {$\mathbf{w_1(z)}$};
            \node [coordinate, right=1cm of w1] (branch) {};
            \node [coordinate, right=1cm of branch] (output) {};
            \node [draw, rectangle, below=1cm of w1, minimum height=2em, minimum width=3em, thick] (w2) {$\mathbf{w_2(z)}$};
            
            \draw [->, thick] (input) -- node [above] {$u_1(z)$} (sum);
            \draw [->, thick] (sum) -- (w1);
            \draw [->, thick] (w1) -- (output) node [above, pos=0.8] {$y_1(z)$};
            \draw [->, thick] (branch) |- (w2);
            \draw [->, thick] (w2) -| (sum);
            
            \fill (branch) circle (2pt);
        \end{tikzpicture}
    \end{figure}
    \begin{itemize}
        \item a negatív visszacsatoláshoz hasonlóan levezethető, az eredmény:
        $$ \frac{y_1(z)}{u_1(z)} = \frac{w_1(z)}{1 + w_1(z)w_2(z)} $$
    \end{itemize}
    \item elágazási pont áthelyezése a tag mögé:
    \begin{figure}[H]
        \centering
        \includegraphics[width=0.8\textwidth]{res/imgs/mecha_8_set1_1.png}
    \end{figure}
    \item elágazási pont áthelyezése a tag elé:
    \begin{figure}[H]
        \centering
        \includegraphics[width=0.8\textwidth]{res/imgs/mecha_8_set1_2.png}
    \end{figure}
    \item összegzési pont áthelyezése a tag mögé:
    \begin{figure}[H]
        \centering
        \includegraphics[width=0.8\textwidth]{res/imgs/mecha_8_set1_3.png}
    \end{figure}
    \item összegzési pont áthelyezése a tag elé:
    \begin{figure}[H]
        \centering
        \includegraphics[width=0.8\textwidth]{res/imgs/mecha_8_set2_1.png}
    \end{figure}
    \item elágazási pont áthelyezése összegzési pont elé:
    \begin{figure}[H]
        \centering
        \includegraphics[width=0.8\textwidth]{res/imgs/mecha_8_set2_2.png}
    \end{figure}
    \item összegzési pont áthelyezése elágazási pont elé:
    \begin{figure}[H]
        \centering
        \includegraphics[width=0.8\textwidth]{res/imgs/mecha_8_set2_3.png}
    \end{figure}
\end{itemize}