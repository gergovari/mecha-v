\subsection{Írja fel a diszkrét rendszerek be-kimeneti modellezését reprezentáló ARMA-alakját!}

\noindent \textbf{ARMA-alak:}
\begin{itemize}
    \item diszkrét idejű rendszer:
    \begin{itemize}
        \item $u(t)$ és $y(t)$ csak diszkrét $t = T_k, k \in \mathbb{N}, t \in \{..., T_{-k}, ..., T_{-1}, T_0, T_1, ..., T_k, ...\}$ időpontok sorozatában van értelmezve, ahol: $T_{k-1} < T_k < T_{k+1}$
        \item általában $T_k = k \cdot T_s$, ahol $T_s$ a mintavételezési idő
        \item $u[k] := u[T_k]$
        \item $y[k] := y[T_k]$
    \end{itemize}
    \item AutoRegresszív Mozgó Átlag
    \item AR – az $y[k]$ kimenő jel korábbi értékei hogyan hatnak vissza a kimenő jel aktuális értékeire
    \item MA – az $u[k]$ bemenő jel korábbi értékei (mozgó átlaga) milyen hatással bírnak az aktuális kimenetre
    \item mi a bemenő és kimenő jel összefüggésének felírására használjuk
    \item előnyös, mert közvetlenül látszik a memória jelleg, mely léptető (shift) memóriával valósítható meg
    \item általános alak:
    $$ \sum_{i=0}^{n} a_{di} y[k-i] = \sum_{i=0}^{r} b_{di} u[k-i] $$
    \item ezt átrendezve a kimenet új értékének számítása:
    $$ y[k] = \sum_{i=0}^{r} b_{di} u[k-i] - \sum_{i=1}^{n} a_{di} y[k-i] \text{, ahol } a_{d0} = 1 \text{: az } y[k] \text{ együtthatója} $$
\end{itemize}

\begin{figure}[H]
    \centering
    \begin{tikzpicture}[auto, node distance=1.5cm, >=latex, thick]
        % Input u[k]
        \node [coordinate] (input) {};
        \node [coordinate, right=1.5cm of input] (split_u) {};
        
        % MA subsystem (top part)
        \node [draw, rectangle, right=1.5cm of split_u] (bd0) {$b_{d0}$};
        
        \node [draw, rectangle, below=0.8cm of split_u] (uk1) {$u[k-1]$};
        \node [draw, rectangle, below=0.1cm of uk1] (uk2) {$u[k-2]$};
        \node [below=0.1cm of uk2] (ukdots) {$\vdots$};
        \node [draw, rectangle, below=0.1cm of ukdots] (ukr) {$u[k-r]$};
        
        \node [draw, rectangle, right=1.5cm of uk1] (bd1) {$b_{d1}$};
        \node [draw, rectangle, right=1.5cm of uk2] (bd2) {$b_{d2}$};
        \node [draw, rectangle, right=1.5cm of ukr] (bdr) {$b_{dr}$};
        
        % Summation MA
        \node [draw, rectangle, right=2cm of bd1, minimum height=3cm] (sumMA) {$\Sigma$};
        
        % Arrows MA
        \draw [->] (input) -- node [above] {$u[k]$} (split_u) -- (bd0);
        \draw [->] (split_u) -- (uk1);
        \draw [->] (uk1) -- (bd1);
        \draw [->] (uk2) -- (bd2);
        \draw [->] (ukr) -- (bdr);
        
        \draw [->] (bd0) -| (sumMA.130);
        \draw [->] (bd1) -- (sumMA.160);
        \draw [->] (bd2) -- (sumMA.185);
        \draw [->] (bdr) -- (sumMA.230);
        
        % Output Summation
        \node [draw, rectangle, right=1cm of sumMA, minimum height=1cm] (sumOut) {$\Sigma$};
        \node [coordinate, right=1.5cm of sumOut] (output) {};
        
        \draw [->] (sumMA) -- (sumOut);
        \draw [->] (sumOut) -- node [above] {$y[k]$} (output);
        
        % AR subsystem (bottom part)
        \node [coordinate, right=0.5cm of sumOut] (split_y) {};
        
        \node [draw, rectangle, below=3.5cm of uk1] (yk1) {$y[k-1]$};
        \node [draw, rectangle, below=0.1cm of yk1] (yk2) {$y[k-2]$};
        \node [below=0.1cm of yk2] (ykdots) {$\vdots$};
        \node [draw, rectangle, below=0.1cm of ykdots] (ykn) {$y[k-n]$};
        
        \node [draw, rectangle, right=1.5cm of yk1] (ad1) {$a_{d1}$};
        \node [draw, rectangle, right=1.5cm of yk2] (ad2) {$a_{d2}$};
        \node [draw, rectangle, right=1.5cm of ykn] (adn) {$a_{dn}$};
        
        % Summation AR (negative)
        \node [draw, rectangle, right=2cm of ad1, minimum height=3cm] (sumAR) {$-\Sigma$};
        
        % Arrows AR
        \draw [->] (output) -- ++(0, -7.5) -| (yk1); % Feedback loop approximate placement
        \path (output) -- ++(0, -7.5) coordinate (feedback_bottom);
        \draw [->] (feedback_bottom) -| (yk1); % Rerouting properly
        % Let's try a cleaner feedback path
        \draw [->] (split_y) |- ($(ykn.south) + (0, -0.5)$) -- (ykn.south); % Not quite right, need to feed into shift register top
        
        % Redoing feedback path
        \draw [->] (split_y) |- ($(yk1.north) + (0, 0.5)$) -- (yk1.north);

        \draw [->] (yk1) -- (ad1);
        \draw [->] (yk2) -- (ad2);
        \draw [->] (ykn) -- (adn);
        
        \draw [->] (ad1) -- (sumAR.160);
        \draw [->] (ad2) -- (sumAR.185);
        \draw [->] (adn) -- (sumAR.230);
        
        \draw [->] (sumAR) -| (sumOut);
        
        % Labels
        \node [above right=0.1cm of bd0] {MA alrendszer};
        \node [below right=0.1cm of adn] {AR alrendszer};
        \node [below=0.1cm of ykn] {léptető memória};
        \node [above=0.1cm of sumMA] {összegző};
        \node [below=0.1cm of sumAR] {összegző};

    \end{tikzpicture}
\end{figure}