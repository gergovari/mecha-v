\subsection{Egy adott, tanult példa (egyenáramú motor) kapcsán ismertesse a struktúra gráf és az impedancia hálózat felrajzolásának lépéseit. Milyen feltételek teljesülése esetén és hogyan lehet csatolt kétpólus elemmel összekapcsolt rendszereket egy oldalra redukálni? Válaszában térjen ki a rendszerek közötti átjárásokat biztosító fizikai összefüggésekre is!}

\noindent \textbf{Egyenáramú motor:}
\begin{itemize}
    \item villamos energiát mechanikus energiává képes alakítani, vagy fordítva: generátor
    \item leggyakrabban állómágnest tartalmaz, az ez által létrehozott mezőben: tekercselt forgórész
\end{itemize}

\begin{figure}[H]
    \centering
    \includegraphics[width=0.45\textwidth]{res/imgs/mecha_23_motor_construction.png}
\end{figure}

\begin{itemize}
    \item a tekercs rendelkezik $R$ ellenállással, illetve $L$ induktivitással, így a villamos hálózattal a következő módon írható le a rendszer:
\end{itemize}

\begin{figure}[H]
    \centering
    \includegraphics[width=0.45\textwidth]{res/imgs/mecha_23_elec_circuit.png}
\end{figure}

\begin{itemize}
    \item a huroktörvényt alkalmazva felírható a rendszerre a következő egyenlet:
    $$ u_{\text{be}}(t) = R i(t) + L \frac{di(t)}{dt} + u_{\text{ind}}(t) $$
    \item a fizikai összefüggésekből levezethető, hogy:
    $$ u_{\text{ind}} = k_e \omega(t), \quad \text{valamint} \quad M_{\text{vill}}(t) = k_m i(t) $$
    \begin{itemize}
        \item ahol $k_e$ a motor sebességállandója, $k_m$ pedig a nyomatékállandó
        \item a két mennyiség értéke SI-ben azonos
    \end{itemize}
    \item írjuk fel a motorra a dinamika alaptételét:
    $$ J \frac{d\omega(t)}{dt} = M_{\text{vill}}(t) - B \omega(t) - M_t(t) $$
    \begin{itemize}
        \item $J$: a motor tehetetlenségi nyomatéka
        \item $B$: viszkózus csillapítási tényező
        \item $M_t(t)$: terhelő nyomaték
    \end{itemize}
\end{itemize}

\noindent \textbf{DC motor struktúragráfja és impedanciahálózata:}

\begin{figure}[H]
    \centering
    \includegraphics[width=0.55\textwidth]{res/imgs/mecha_23_structural_graph.png}
\end{figure}

\begin{itemize}
    \item $M_t$ nyomaték előjele negatív, mivel a terhelés csökkenti a fordulatszámot
\end{itemize}

\begin{figure}[H]
    \centering
    \includegraphics[width=0.55\textwidth]{res/imgs/mecha_23_impedance_network.png}
\end{figure}

\begin{itemize}
    \item az impedanciák értékei (Laplace tartományban):
    \begin{itemize}
        \item $Z_R = R$
        \item $Z_L = sL$
        \item $Z_B = \frac{1}{B}$
        \item $Z_J = \frac{1}{Js}$
    \end{itemize}
    \item akkor lehet a kapcsolt kétpólus oldalait redukálni, ha:
    \begin{itemize}
        \item veszteségmentes: $P_{\text{be}} = P_{\text{ki}}$, előjelkonvenció miatt $P_{\text{be}} + P_{\text{ki}} = 0$
        \item lineáris, statikus rendszer -- algebrai egyenletekkel leírható
        \item a transzformátor egyenletei:
        $$ \begin{bmatrix} \chi_{12} \\ \phi_1 \end{bmatrix} = \begin{bmatrix} c_{11} & 0 \\ 0 & -\frac{1}{c_{11}} \end{bmatrix} \begin{bmatrix} \chi_{34} \\ \phi_2 \end{bmatrix} \implies \begin{bmatrix} u_{\text{ind}} \\ i \end{bmatrix} = \begin{bmatrix} k_e & 0 \\ 0 & -\frac{1}{k_m} \end{bmatrix} \begin{bmatrix} \omega \\ M_{\text{vill}} \end{bmatrix} $$
        \item impedanciák:
        $$ Z_{\text{mech}} = \frac{\Omega(s)}{M(s)} = \frac{\frac{1}{k_e} U(s)}{k_m I(s)} = \frac{1}{k_e k_m} \frac{U(s)}{I(s)} = \frac{1}{k_e k_m} Z_{\text{vill}} $$
    \end{itemize}
    \item a soros és párhuzamos impedanciákat össze tudjuk vonni:
    \begin{itemize}
        \item $Z_{e, RL} = R + sL$
        \item $Z_{e, BJ} = \frac{1}{B + Js}$
    \end{itemize}
    \item redukálás:
\end{itemize}

\begin{figure}[H]
    \centering
    \includegraphics[width=0.45\textwidth]{res/imgs/mecha_23_reduced_network.png}
\end{figure}

\begin{itemize}
    \item innen már a tanult módszerek segítségével tudunk számolni
\end{itemize}

\noindent \textbf{Fizikai összefüggések:}
\begin{itemize}
    \item \textbf{indukált feszültség:}
    \begin{itemize}
        \item a mágneses fluxus változásából írhatjuk le - \textbf{Faraday törvény:}
        $$ u_{\text{ind}}(t) = \frac{d\Phi(t)}{dt} = B_x \frac{dA(t)}{dt} = B_x \frac{d(l \cdot 2r \sin\varphi(t))}{dt} = (B_x l \cdot 2r \cos\varphi(t)) \omega(t) $$
        \item az indukált feszültséget és a szögsebességet összekötő tagot elnevezzük sebességállandónak, jele: $k_e$
    \end{itemize}
    \item \textbf{motor villamos nyomatéka:}
    \begin{itemize}
        \item a Lorentz-erőből adódó forgatónyomatékkal számolható
        \item arányos az áramerősséggel
    \end{itemize}
\end{itemize}

\begin{figure}[H]
    \centering
    \includegraphics[width=0.6\textwidth]{res/imgs/mecha_23_lorentz_torque.png}
\end{figure}

\begin{itemize}
    \item a Lorentz-erő:
    $$ \underline{F} = i(\underline{l} \times \underline{B}) = i \begin{bmatrix} 0 \\ 0 \\ -l \end{bmatrix} \times \begin{bmatrix} B_x \\ 0 \\ 0 \end{bmatrix} = \begin{bmatrix} 0 \\ -iB_x l \\ 0 \end{bmatrix} $$
    \item a Lorentz-erőből számolt nyomaték:
    $$ \underline{M} = \underline{r} \times \underline{F} = \begin{bmatrix} r \cos\varphi \\ r \sin\varphi \\ 0 \end{bmatrix} \times \begin{bmatrix} 0 \\ F \\ 0 \end{bmatrix} = \begin{bmatrix} 0 \\ 0 \\ r F \cos\varphi \end{bmatrix} $$
    \item $M_{\text{vill}} = 2M = (2 B_x l \cdot r \cos\varphi(t)) i$
    \item az áramot és a nyomatékot összekötő tagot elnevezzük nyomatékállandónak, jele: $k_m$
\end{itemize}