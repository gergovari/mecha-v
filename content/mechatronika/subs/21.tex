\subsection{Adja meg az villamos rendszer (kapcsolt elektromechanikai), haladó és forgómozgású mechanikai rendszerek és az áramlástechnikai (pneumatikus és hidraulikai) rendszerek koncentrált paraméterű leírása esetén az átmenő és keresztváltozó típusát, valamint az energiatárolókat (amennyiben léteznek) és disszipatív elemeket.}

\noindent \textbf{Villamos rendszer:}
\begin{itemize}
    \item $\phi$: $i$ -- áramerősség [A]
    \item $\chi$: $u$ -- feszültség [V]
    \item energiatárolók:
    \begin{itemize}
        \item[] $ E_\chi = \int Q du = \int C u du = \frac{1}{2} C u^2 $
        \item[] $ E_\phi = \int i d\phi_m = \int \frac{1}{L} \phi_m d\phi_m = \frac{1}{2L} \phi_m^2 = \frac{1}{2} L i^2 $
    \end{itemize}
    \item disszipatív elem:
    \begin{itemize}
        \item ellenállás
        \item $i = \frac{1}{R} u$
        \item $[R] = \Omega$
    \end{itemize}
    \item további elemek:
    \begin{itemize}
        \item L -- induktivitás [H]
        \item C -- kapacitás [F]
        \item $\phi_m$ -- mágneses fluxus [Wb] = [Vs]
        \item Q -- töltés [C] = [As]
    \end{itemize}
\end{itemize}

\begin{figure}[H]
    \centering
    \includegraphics[width=0.7\textwidth]{res/imgs/mecha_21_elec_tetrahedron.png}
\end{figure}

\newpage
\noindent \textbf{Mechanikai haladó:}
\begin{itemize}
    \item $\phi$: $f$ -- erő [N]
    \item $\chi$: $v$ -- sebesség [m/s]
    \item energiatárolók:
    \begin{itemize}
        \item[] $ E_\chi = \int I dv = \int m v dv = \frac{1}{2} m v^2 $
        \item[] $ E_\phi = \int f dx = \int kx dx = \frac{1}{2} k x^2 $
    \end{itemize}
    \item disszipatív elem:
    \begin{itemize}
        \item viszkózus csillapítási tényező
        \item $f = bv$
        \item $[b] = \text{Ns/m}$
    \end{itemize}
    \item további elemek:
    \begin{itemize}
        \item I -- lendület [Ns] = [kg m/s]
        \item x -- elmozdulás [m]
        \item k -- rugómerevség [N/m]
        \item m -- tömeg [kg]
    \end{itemize}
\end{itemize}

\begin{figure}[H]
    \centering
    \includegraphics[width=0.45\textwidth]{res/imgs/mecha_21_mech_trans_tetrahedron.png}
\end{figure}

\noindent \textbf{Mechanikai forgó:}
\begin{itemize}
    \item $\phi$: $M$ -- nyomaték [Nm]
    \item $\chi$: $\omega$ -- szögsebesség [rad/s]
    \item energiatárolók:
    \begin{itemize}
        \item[] $ E_\chi = \int \pi d\omega = \int J \omega d\omega = \frac{1}{2} J \omega^2 $
        \item[] $ E_\phi = \int M d\varphi = \int k \varphi d\varphi = \frac{1}{2} k \varphi^2 $
    \end{itemize}
    \item disszipatív elem:
    \begin{itemize}
        \item viszkózus csillapítási tényező
        \item $M = B\omega$
        \item $[B] = \text{Nms/rad}$
    \end{itemize}
    \item további elemek:
    \begin{itemize}
        \item $\pi$ -- perdület [Nms]
        \item $\varphi$ -- szögelfordulás [rad]
        \item k -- torziós rugómerevség [Nm/rad]
        \item J -- tehetetlenségi nyomaték [kgm$^2$]
    \end{itemize}
\end{itemize}

\noindent $\dot{\pi} = M$ \hfill $[J] = \text{kgm}^2$ \hfill $[k] = \frac{\text{Nm}}{\text{rad}}$ \hfill \textcolor{red}{3D-ben szögvektor NEM létezik!} \hfill $[\varphi] = \text{rad}$

\begin{figure}[H]
    \centering
    \includegraphics[width=0.45\textwidth]{res/imgs/mecha_21_mech_rot_tetrahedron.png}
\end{figure}

\noindent \textbf{Áramlástechnikai:}
\begin{itemize}
    \item $\phi$: $q_v$ -- tömegáram/térfogatáram
    \item $\chi$: $p_{12}$ -- nyomás
    \item energiatárolók:
    \begin{itemize}
        \item[] fluid kapacitás: $q_v = C_f \frac{dp_{12}}{dt}$
        \item[] fluid induktivitás: $q_v = \int_{T_1}^{T_2} \frac{1}{L_f} p dt$
    \end{itemize}
    \item disszipatív elem:
    \begin{itemize}
        \item fluid ellenállás
        \item $q_v = \frac{1}{R_f} p_{12}$
    \end{itemize}
    \item áramlástechnikai (fluid) rendszerek:
    \begin{itemize}
        \item hidraulikus: nem összenyomható közeg
        \item pneumatikus: összenyomható közeg, légköri nyomásnál nagyobb nyomás
        \item akusztikus: összenyomható közeg, légköri nyomás
    \end{itemize}
\end{itemize}

\begin{figure}[H]
    \centering
    \includegraphics[width=0.35\textwidth]{res/imgs/mecha_21_fluid_tetrahedron.png}
\end{figure}
\noindent nem igazán használt [q?]

\noindent \textbf{hidraulikus rendszer kapacitás:}

\begin{minipage}[c]{0.7\textwidth}
\begin{itemize}
    \item nyitott tartály
    \item tápnyomás: $p_t = \rho g h$
    \item $q_v = \frac{dV}{dt} = A v \to V = A h \to \frac{dV}{dt} = A \frac{dh}{dt}$ \\
    $\frac{dp}{dt} = \rho g \frac{dh}{dt} \to \frac{dp}{dt} = \rho g \frac{1}{A} \frac{dV}{dt} $ \\
    \fbox{$q_v = \frac{A}{\rho g} \frac{dp_{12}}{dt}$} \quad hidraulikus kapacitás $C_f$
\end{itemize}
\end{minipage}
\begin{minipage}[c]{0.28\textwidth}
    \centering
    \includegraphics[width=\textwidth]{res/imgs/mecha_21_pipe.png}
\end{minipage}


\begin{table}[H]
    \centering
    \begin{tabular}{c|c|c}
        & Pneumatikus & Hidraulikus \\
        \hline
        kapacitás & $C_f = \frac{V}{\kappa p_1}$ & $C_f = \frac{A}{\rho g}$ \\
        induktivitás & nincs & $L_f = \frac{\rho l}{A}$ \\
        disszipatív elem & $R_f$ & $R_f$
    \end{tabular}
\end{table}
