\subsection{Adja meg az villamos rendszer (kapcsolt elektromechanikai), haladó és forgómozgású mechanikai rendszerek és az áramlástechnikai (pneumatikus és hidraulikai) rendszerek koncentrált paraméterű leírása esetén az átmenő és keresztváltozó típusát, valamint az energiatárolókat (amennyiben léteznek) és disszipatív elemeket.}

\noindent \textbf{Villamos rendszer:}
\begin{itemize}
    \item $\Phi$: $i$ -- áramerősség [A] (átmenő változó)
    \item $\chi$: $u$ -- feszültség [V] (keresztváltozó)
    \item \textbf{energiatárolók:}
    $$ E_\chi = \int Q du = \int C u du = \frac{1}{2} C u^2 $$
    $$ E_\Phi = \int i d\phi_m = \int \frac{1}{L} \phi_m d\phi_m = \frac{1}{2L} \phi_m^2 = \frac{1}{2} L i^2 $$
    \item \textbf{disszipatív elem:}
    \begin{itemize}
        \item ellenállás
        \item $i = \frac{1}{R} u$
        \item $[R] = \Omega$
    \end{itemize}
    \item \textbf{további elemek:}
    \begin{itemize}
        \item $L$ -- induktivitás [H]
        \item $C$ -- kapacitás [F]
        \item $\phi_m$ -- mágneses fluxus [Wb] = [Vs]
        \item $Q$ -- töltés [C] = [As]
    \end{itemize}
\end{itemize}

\begin{figure}[H]
    \centering
    \includegraphics[width=0.5\textwidth]{res/imgs/mecha_21_elec_tetrahedron.png}
\end{figure}

\newpage
\noindent \textbf{Mechanikai haladó:}
\begin{itemize}
    \item $\Phi$: $f$ -- erő [N]
    \item $\chi$: $v$ -- sebesség [m/s]
    \item \textbf{energiatárolók:}
    $$ E_\chi = \int I dv = \int m v dv = \frac{1}{2} m v^2 $$
    $$ E_\Phi = \int f dx = \int kx dx = \frac{1}{2} k x^2 $$
    \item \textbf{disszipatív elem:}
    \begin{itemize}
        \item viszkózus csillapítási tényező
        \item $f = bv$
        \item $[b] = \text{Ns/m}$
    \end{itemize}
    \item \textbf{további elemek:}
    \begin{itemize}
        \item $I$ -- lendület [Ns] = [kg m/s]
        \item $x$ -- elmozdulás [m]
        \item $k$ -- rugómerevség [N/m]
        \item $m$ -- tömeg [kg]
    \end{itemize}
\end{itemize}

\begin{figure}[H]
    \centering
    \includegraphics[width=0.5\textwidth]{res/imgs/mecha_21_mech_trans_tetrahedron.png}
\end{figure}

\noindent \textbf{Mechanikai forgó:}
\begin{itemize}
    \item $\Phi$: $M$ -- nyomaték [Nm]
    \item $\chi$: $\omega$ -- szögsebesség [rad/s]
    \item \textbf{energiatárolók:}
    $$ E_\chi = \int \pi d\omega = \int J \omega d\omega = \frac{1}{2} J \omega^2 $$
    $$ E_\Phi = \int M d\varphi = \int k \varphi d\varphi = \frac{1}{2} k \varphi^2 $$
    \item \textbf{disszipatív elem:}
    \begin{itemize}
        \item viszkózus csillapítási tényező
        \item $M = B\omega$
        \item $[B] = \text{Nms/rad}$
    \end{itemize}
    \item \textbf{további elemek:}
    \begin{itemize}
        \item $\pi$ -- perdület [Nms]
        \item $\varphi$ -- szögelfordulás [rad]
        \item $k$ -- torziós rugómerevség [Nm/rad]
    \end{itemize}
\end{itemize}