\subsection{Ismertesse az egytárolós arányos tag súly és átmeneti függvényeit! Válaszában térjen ki az időállandó fogalmára!}

\noindent \textbf{Arányos egytárolós tag -- PT1:}
$$ W(s) = \frac{A}{s-p_1} = \frac{K}{1+sT} $$

\noindent \textbf{Arányos egytárolós tag súlyfüggvénye:}
\begin{itemize}
    \item az egységimpulzusra ($\delta(t)$-re) adott válasz
    \item $u(t) = \delta(t) \rightarrow U(s) = 1$
    $$ Y(s) = W(s)U(s) = \frac{K}{1+sT} \cdot 1 $$
    \item a kérdés: $y(t) = w(t) = ?$
    $$ y(t) = w(t) = \frac{K}{T} \mathcal{L}^{-1} \left\{ \frac{1}{s + \frac{1}{T}} \right\} = \frac{K}{T} e^{-\frac{t}{T}} \varepsilon(t) $$
    \item grafikusan ábrázolva:
    \begin{figure}[H]
        \centering
        \includegraphics[width=0.4\textwidth]{res/imgs/mecha_13_weight_func.png}
    \end{figure}
\end{itemize}

\noindent \textbf{Arányos egytárolós tag átmeneti függvénye:}
\begin{itemize}
    \item egységugrásra ($\varepsilon(t)$-re) adott válasz
    \item $u(t) = \varepsilon(t) \rightarrow U(s) = 1/s$
    $$ Y(s) = W(s)U(s) = \frac{K}{1+sT} \frac{1}{s} $$
    \item a kérdés: $y(t) = v(t) = ?$
    $$ y(t) = v(t) = \mathcal{L}^{-1} \left\{ \frac{K}{(1+sT)s} \right\} = K \mathcal{L}^{-1} \left\{ \frac{1}{T(s+\frac{1}{T})s} \right\} = \frac{K}{T} \mathcal{L}^{-1} \left\{ \frac{A}{s} + \frac{B}{s + \frac{1}{T}} \right\} $$
    $$ A = T, \quad B = -T $$
    $$ v(t) = K \mathcal{L}^{-1} \left\{ \frac{1}{s} - \frac{1}{s + \frac{1}{T}} \right\} = K \left( 1 - e^{-\frac{t}{T}} \right) \varepsilon(t) $$
    \item grafikusan ábrázolva:
    \begin{figure}[H]
        \centering
        \includegraphics[width=0.4\textwidth]{res/imgs/mecha_13_step_func.png}
    \end{figure}
\end{itemize}
\newpage
\noindent \textbf{T és K változók az átmeneti függvénynél:}
\begin{itemize}
    \item K meghatározza, hogy hova konvergál a rendszer
    \item T meghatározza, hogy mennyi idő alatt konvergál
\end{itemize}

\noindent \textbf{Időállandó fogalma:}
\begin{itemize}
    \item Az átviteli függvény időállandós alakjából s együtthatói, melyet az átviteli függvény gyöktényezős alakjából tudunk átalakítani
    \item így az időállandó(k) a gyöktényezős alak – mely s-re nézve egy valós együtthatójú racionális törtfüggvény – \textbf{pólusai} valós részének reciprokának mínusz egyszerese:
    $$ W(s) = \frac{b_r}{a_n} \frac{(s-z_1)(s-z_2) \dots (s-z_r)}{(s-p_1)(s-p_2) \dots (s-p_n)} $$
    $$ W(s) = \underbrace{ \frac{b_r (-z_1)(-z_2) \dots (-z_r)}{a_n (-p_1)(-p_2) \dots (-p_n)} }_{\text{erősítés}} \frac{(1-\frac{1}{z_1}s)(1-\frac{1}{z_2}s) \dots (1-\frac{1}{z_r}s)}{(1-\frac{1}{p_1}s)(1-\frac{1}{p_2}s) \dots (1-\frac{1}{p_n}s)} $$
    \item időállandók: $T_1 = -\frac{1}{p_1}, T_2 = -\frac{1}{p_2}, \dots, T_n = -\frac{1}{p_n}$
\end{itemize}