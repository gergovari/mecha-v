\subsection{Ismertesse a frekvencia-átviteli függvény fogalmát, illetve annak megjelenítési módjait (Bode diagram)!}

\noindent \textbf{A kérdés, hogy egy folytonos idejű időinvariáns rendszer milyen választ ad harmonikus gerjesztésre:}
\begin{itemize}
    \item szinuszos vizsgáló jelet használunk:
    $$ u(t) = A \sin(\omega t) \to U(s) = \frac{A\omega}{s^2 + \omega^2} $$
    \item harmonikus gerjesztés esetén a válasz (állandósult értéke):
    $$ y(t) = \tilde{A}(\omega) \sin(\omega t - \varphi(\omega)) $$
    $\tilde{A}(\omega)$: frekvenciafüggő amplitúdó, \quad $\varphi(\omega)$: frekvenciafüggő fáziseltolás
    \item formális helyettesítés: $s \to j\omega$ komplex szám
    \begin{itemize}
        \item \textbf{frekvencia-átviteli függvény az átviteli függvényből:}
        $$ W(j\omega) = W(s)|_{s=j\omega} $$
    \end{itemize}
    \item a komplex szám nagysága:
    $$ |W(j\omega)| = \frac{A_{ki}}{A_{be}} = A_r, \quad \text{ahol } A_{ki} = \tilde{A}, \ A_{be} = A, \ A_r: \text{amplitúdó arány} $$
    \item fáziskülönbség:
    $$ \Delta \varphi = \varphi_{ki} - \varphi_{be} = \arg(W(j\omega)) = \arctan \left( \frac{\text{Im}\{W(j\omega)\}}{\text{Re}\{W(j\omega)\}} \right) $$
\end{itemize}

\noindent \textbf{Frekvencia-átviteli függvény megjelenítése:}
\begin{itemize}
    \item \textbf{Nyquist-diagram:}
    \begin{figure}[H]
        \centering
        \includegraphics[width=0.4\textwidth]{res/imgs/mecha_17_nyquist.png}
    \end{figure}
    \item \textbf{Bode diagram(ok):}
    \begin{itemize}
        \item[a)] amplitúdó -- körfrekvencia diagram (teljesítmény arány)
        \item[b)] fázisszög -- körfrekvencia diagram (fáziskülönbség)
    \end{itemize}
\end{itemize}

\noindent \textbf{Amplitúdó -- körfrekvencia diagram:}
$$ A_r \to P_r = A_r^2 $$
$$ B = \lg(P_r) = \lg(A_r^2) \ [B] \text{ (bel)} $$
$$ B = 10 \lg(P_r) \ [dB] \text{ (decibel)} $$
$$ B = 10 \lg(A_r^2) = 20 \lg(A_r) \ [dB] $$

\noindent \textbf{Diagramok:}
\begin{figure}[H]
    \centering
    \includegraphics[width=0.6\textwidth]{res/imgs/mecha_17_bode_axis.png}
\end{figure}

\begin{itemize}
    \item a frekvencia-átviteli függvény szorzattagokra bontható
    \item a diagram ezeknek a tagoknak a diagramjának az összegzésével kapható meg, bizonyítás:
    \begin{itemize}
        \item komplex szám másik alakja: $W(j\omega) = re^{i\varphi}$
        \item szorzatokra bontható: $W(j\omega) = W_1(j\omega)W_2(j\omega)$
        \item $W(j\omega) = r_1 e^{i\varphi_1} r_2 e^{i\varphi_2} = r_1 r_2 e^{i(\varphi_1 + \varphi_2)}$
        \item $\lg(r_1 r_2) = \lg(r_1) + \lg(r_2)$
    \end{itemize}
\end{itemize}

\noindent \textbf{Az alapelemek (általános alak):}
\begin{figure}[H]
    \centering
    \includegraphics[width=0.8\textwidth]{res/imgs/mecha_17_general_tf.png}
\end{figure}

\begin{itemize}
    \item \textbf{Arányos tag:} $W(s) = K$
    \begin{figure}[H]
        \centering
        \includegraphics[width=0.3\textwidth]{res/imgs/mecha_17_prop_tag.png}
    \end{figure}

    \item \textbf{Deriváló tag:} $W(s) = s \to W(j\omega) = j\omega$
    \begin{figure}[H]
        \centering
        \includegraphics[width=0.3\textwidth]{res/imgs/mecha_17_deriv_tag.png}
    \end{figure}

    \item \textbf{integráló tag:} $W(s) = \frac{1}{s} \to W(j\omega) = \frac{1}{j\omega} = -j\frac{1}{\omega}$
    \begin{figure}[H]
        \centering
        \includegraphics[width=0.3\textwidth]{res/imgs/mecha_17_integ_tag.png}
    \end{figure}

    \item \textbf{tiszta pólus:} $W(s) = \frac{1}{1+sT} \to W(j\omega) = \frac{1-j\omega T}{1+\omega^2 T^2}$
    \begin{figure}[H]
        \centering
        \includegraphics[width=0.6\textwidth]{res/imgs/mecha_17_real_pole.png}
    \end{figure}

    \item \textbf{tiszta zérus:} $W(s) = 1 + sT \to W(j\omega) = 1 + j\omega T$
    \begin{itemize}
        \item tiszta pólus esetén kapott eredmények tükrözése y tengelyre
    \end{itemize}

    \item \textbf{komplex póluspár:} $W(s) = \frac{\omega_n^2}{s^2+2\zeta\omega_n s + \omega_n^2} \to W(j\omega) = \frac{\omega_n^2}{-\omega^2+2\zeta\omega_n \omega j + \omega_n^2}$
    $$ W(j\omega) = \frac{1}{1 - \frac{\omega^2}{\omega_n^2} + \frac{2\zeta\omega j}{\omega_n}} $$
    \begin{itemize}
        \item frekvenciahányados: $\lambda = \frac{\omega}{\omega_n}$
    \end{itemize}
    $$ W(j\omega) = \frac{1 - \lambda^2 - 2\zeta\lambda j}{((1 - \lambda^2)^2 + 4\zeta^2\lambda^2)^2} $$
    \begin{figure}[H]
        \centering
        \includegraphics[width=0.6\textwidth]{res/imgs/mecha_17_complex_pole.png}
    \end{figure}
\end{itemize}