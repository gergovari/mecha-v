\subsection{Egy adott, tanult példa (fogaskerékhajtómű, fogaskerék-fogasléc) kapcsán ismertesse a struktúra gráf és az impedancia hálózat felrajzolásának lépéseit. Milyen feltételek teljesülése esetén és hogyan lehet csatolt kétpólus elemmel összekapcsolt rendszereket egy oldalra redukálni? Válaszában térjen ki a rendszerek közötti átjárásokat biztosító fizikai összefüggésekre is!}

\noindent \textbf{Fogaskerék-hajtómű:}

\begin{figure}[H]
    \centering
    \includegraphics[width=0.6\textwidth]{res/imgs/mecha_24_gear_drive_construction.png}
\end{figure}

\begin{itemize}
    \item \textbf{struktúragráf:}
\end{itemize}

\begin{figure}[H]
    \centering
    \includegraphics[width=0.45\textwidth]{res/imgs/mecha_24_gear_drive_graph.png}
\end{figure}

\begin{itemize}
    \item akkor lehet a kapcsolt kétpólus oldalait redukálni, ha:
    \begin{itemize}
        \item veszteségmentes: $P_{\text{be}} = P_{\text{ki}}$, előjelkonvenció miatt $P_{\text{be}} + P_{\text{ki}} = 0$
        \item lineáris, statikus rendszer - algebrai egyenletekkel leírható
    \end{itemize}
    \item \textbf{fizikai összefüggések a kapcsolóegyenlethez:}
    \begin{itemize}
        \item $d\pi = pz \to d = \frac{p}{\pi}z$
        \item a kerületi sebességek: $v_T = \Omega_1 r_1 = \Omega_2 r_2 \to \Omega_1 = \frac{r_2}{r_1}\Omega_2 = \frac{z_2}{z_1}\Omega_2 = i \Omega_2$
        \item nyomaték: $M = fr \to \left. \begin{matrix} M_1 = f_T r_1 \\ M_2 = -f_T r_2 \end{matrix} \right\} M_1 = -\frac{r_1}{r_2} M_2 = -\frac{1}{i} M_2$
    \end{itemize}
\end{itemize}

\begin{figure}[H]
    \centering
    \includegraphics[width=0.7\textwidth]{res/imgs/mecha_24_gear_geometry.png}
\end{figure}

\begin{itemize}
    \item ezek lesznek a kapcsolóegyenleteink, így az impedanciahálózat:
\end{itemize}

\begin{figure}[H]
    \centering
    \includegraphics[width=0.7\textwidth]{res/imgs/mecha_24_gear_drive_impedance.png}
\end{figure}

\begin{itemize}
    \item \textbf{redukált impedanciák:}
    $$ Z_{\text{forg1}} = \frac{\Omega_1}{M_1} = \frac{i \Omega_2}{\frac{1}{i} M_2} = i^2 \frac{\Omega_2}{M_2} = i^2 Z_{\text{forg2}} $$
\end{itemize}

\noindent \textbf{Fogaskerék -- fogasléc:}

\begin{figure}[H]
    \centering
    \includegraphics[width=0.45\textwidth]{res/imgs/mecha_24_rack_pinion_construction.png}
\end{figure}

\begin{itemize}
    \item \textbf{struktúragráf:}
\end{itemize}

% [Diagram: Fogaskerék-fogasléc struktúragráfja]

\begin{itemize}
    \item \textbf{fizikai összefüggés:}
    $$ \Omega_g = \frac{1}{r} v_T, \quad M = -r f_T $$
    \item \textbf{így az impedanciahálózatunk:}
\end{itemize}

% [Diagram: Fogaskerék-fogasléc impedanciahálózata]