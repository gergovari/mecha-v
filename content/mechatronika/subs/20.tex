\subsection{Ismertesse a következő fogalmakat (adja meg a definícióját és rövid értelmezését): extenzív és intenzív fizikai mennyiségek, átmenő és keresztváltozók, energiatárolók (átmenő és keresztváltozóval) és disszipatív elemek (kétpólusok), csatolt kétpólus elem (transzformátor és girátor)!}

\begin{itemize}
    \item \textbf{extenzív fizikai mennyiség:} olyan mennyiség, melynek a rendszer egészét jellemző értéke megegyezik az őt alkotó részrendszereket jellemző értékek összegével (pl. tömeg, térfogat, töltés).
    \item \textbf{intenzív fizikai mennyiség:} a rendszer egészét jellemző értéke egyensúly esetén megegyezik a rendszert alkotó részrendszerek értékeivel (pl. hőmérséklet, nyomás).
    \item \textbf{átmenő változók ($\Phi$):} extenzív fizikai mennyiség rátája ($d/dt$), általában megmaradási törvény is tartozik hozzá.
    \item \textbf{keresztváltozók ($\chi$):} intenzív fizikai mennyiségek (különbsége) vagy általános elmozdulás ($q$) rátája.
\end{itemize}

\noindent \textbf{Állapot-tetraéder (Tetrahedron of state):}
\begin{figure}[H]
    \centering
    \includegraphics[width=0.4 \textwidth]{res/imgs/mecha_20_tetrahedron.png}
\end{figure}

\noindent \textbf{A rendszerelemek feloszthatók energetikai szempontból:}
\begin{itemize}
    \item \textbf{energiatárolók:} energiát tárolnak, az elem valamilyen extenzív mennyiség energiáját halmozza fel, lehet kapacitív vagy induktív.
    \item \textbf{disszipatív elemek:} veszteséget modellezik, a valós rendszerek modellezéséhez szükségesek.
    \item \textbf{energiaátalakítók:} csatolt kétpólus, négypólus elem.
    \item \textbf{energiaforrások}.
\end{itemize}

\noindent \textbf{Energiaátalakítók:}
\begin{figure}[H]
    \centering
    \includegraphics[width=0.4 \textwidth]{res/imgs/mecha_20_converter.png}
\end{figure}

\begin{itemize}
    \item \textbf{veszteségmentes:} $P_{be} = P_{ki}$, előjelkonvenció miatt $P_{be} + P_{ki} = 0 \implies \chi_{be}\Phi_{be} + \chi_{ki}\Phi_{ki} = 0$
    \item \textbf{lineáris, statikus rendszer} $\rightarrow$ algebrai egyenlet:
    $$ \begin{bmatrix} \chi_{be} \\ \Phi_{be} \end{bmatrix} = \begin{bmatrix} c_{11} & c_{12} \\ c_{21} & c_{22} \end{bmatrix} \begin{bmatrix} \chi_{ki} \\ \Phi_{ki} \end{bmatrix} $$
    $$ \chi_{be} = c_{11}\chi_{ki} + c_{12}\Phi_{ki} $$
    $$ \Phi_{be} = c_{21}\chi_{ki} + c_{22}\Phi_{ki} $$
    \item \textbf{behelyettesítve a teljesítmény-egyenletbe:}
    $$ (c_{11}\chi_{ki} + c_{12}\Phi_{ki})(c_{21}\chi_{ki} + c_{22}\Phi_{ki}) + \chi_{ki}\Phi_{ki} = 0 $$
    $$ c_{11}c_{21}\chi_{ki}^2 + (1 + c_{11}c_{22} + c_{12}c_{21})\chi_{ki}\Phi_{ki} + c_{12}c_{22}\Phi_{ki}^2 = 0 $$
    \item \textbf{2 nem triviális megoldás:}
\end{itemize}

\noindent \textbf{1. lehetőség:} $c_{12} = c_{21} = 0$, ekkor $(1 + c_{11}c_{22}) = 0 \implies c_{22} = - \frac{1}{c_{11}}$
\begin{itemize}
    \item energiaátalakító: \textbf{transzformátor}
    \item azonos típusú változók között teremt kapcsolatot (pl. villamos transzformátor, hatómű, DC motor)
    $$ \begin{bmatrix} \chi_{be} \\ \Phi_{be} \end{bmatrix} = \begin{bmatrix} c_{11} & 0 \\ 0 & -\frac{1}{c_{11}} \end{bmatrix} \begin{bmatrix} \chi_{ki} \\ \Phi_{ki} \end{bmatrix} $$
\end{itemize}

\noindent \textbf{2. lehetőség:} $c_{11} = c_{22} = 0$, ekkor $(1 + c_{12}c_{21}) = 0 \implies c_{21} = - \frac{1}{c_{12}}$
\begin{itemize}
    \item energiaátalakító: fordítóváltó, \textbf{girátor}
    \item eltérő típusú változók között teremt kapcsolatot (pl. giroszkóp, munkahenger)
    $$ \begin{bmatrix} \chi_{be} \\ \Phi_{be} \end{bmatrix} = \begin{bmatrix} 0 & c_{12} \\ -\frac{1}{c_{12}} & 0 \end{bmatrix} \begin{bmatrix} \chi_{ki} \\ \Phi_{ki} \end{bmatrix} $$
\end{itemize}