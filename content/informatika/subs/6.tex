\subsection{Az adatbázisok alapjai. Adatmodellezés. Kapcsolatok típusai. Relációs adatbázismodell. Relációk jellemzői. A relációs algebra műveletei. SQL alapok, lekérdezések.}

\noindent \textbf{Adatbázisok:}
\begin{itemize}
    \item rekord: összetartozó adatok egy példányhoz
    \item adattételekből áll
    \item összetartozó adatokat tárol
    \item \textbf{adatmodell:} a valós világ elemeinek és köztük lévő kapcsolatok leképzése adatokra
    \begin{itemize}
        \item egész pontosan: az azokról meglévő ismereteinket képezzük le az adatokra
    \end{itemize}
    \item szűkebb értelemben: a leírásra szolgáló adatok szerkezete
    \item az adatmodell általánosan tartalmazza:
    \begin{itemize}
        \item az alapelemeket
        \item az alapelemekkel végzett műveleteket
        \item integritási kényszereket:
        \begin{itemize}
            \item adatszerkezet: adatok tárolására szolgáló elemi adatok rendszere
            \item művelet: a megjelenítési igényeknek megfelelően (pl. indexek)
            \item integritási kényszer: ellentmondás mentességet biztosító feltételek, megszorítások
        \end{itemize}
    \end{itemize}
\end{itemize}

\noindent \textbf{Egyed – kapcsolat modell:}
\begin{itemize}
    \item egyed: a világ egy megkülönböztetett objektuma
    \item egyedek egy halmaza – rekordtípus
    \item egyedelőfordulás-halmaz: azonos, megkülönböztető tulajdonsággal jellemzett egyedelőfordulások gyűjteménye
    \item megkülönböztethetőség biztosítása:
    \begin{itemize}
        \item természetes módon (ujjlenyomat)
        \item mesterségesen (OM azonosító, személyi, TAJ, stb)
    \end{itemize}
    \item kapcsolat: egyedek közt fennálló viszony
    \item kapcsolattípus előfordulás halmaz: két egyedelőfordulás halmaz, illetve az egyedelőfordulások között fennálló viszony
    \item kapcsolat típusa:
    \begin{itemize}
        \item 1:1 (házasság)
        \item 1:N (oktató – csoport)
        \item N:M (ingatlan – nyilvántartó)
    \end{itemize}
    \item tulajdonság: egyedek, illetve kapcsolatok azon jellemzői, melyek az egyedtípust definiálják / az egyedtípusok közti kapcsolatok jellemzői (szőke ember)
    \item tulajdonságérték: az egyed-, vagy kapcsolatelőfordulásokhoz tartozó konkrét adat (szőke)
    \item értékhalmaz: a tulajdonság lehetséges értékei (szőke, barna, fekete, vörös)
\end{itemize}

\noindent \textbf{Reláció:}
\begin{itemize}
    \item n darab halmaz direkt szorzatának részhalmaza
    \item névvel azonosítjuk
    \item minden sor különböző benne
    \item kulcs: a reláció bármely elemét egyértelműen azonosítja
    \item fokszám: a relációt alkotó halmazok (értelmezési tartományok) száma
    \item kardinalitás: a reláció elemeinek száma
    \item attribútum: egyes elemekben a tényezők konkrét értéke
    \item az egyes elemeknek nincs sorrendje
    \item \textbf{műveletek:}
    \begin{itemize}
        \item egyesítés
        \item metszet
        \item különbség
    \end{itemize}
    \item \textbf{csonkító műveletek:}
    \begin{itemize}
        \item vetítés (projection): tényezők (értelmezési tartományok) kiemelése
        \item kiválasztás (select): elemek kiválasztása
    \end{itemize}
    \item \textbf{kapcsoló műveletek:}
    \begin{itemize}
        \item join: kapcsoló tényezők kapcsolnak
        \item Descartes szorzat: mindent mindennel (direkt szorzat)
    \end{itemize}
    \item a reláció ábrázolható kétdimenziós adattáblával
    \item oszlopok: tulajdonságok
    \item a táblát névvel azonosítjuk (reláció neve)
    \item sorok: reláció előfordulások halmaza – nincs két azonos sor
    \item tulajdonságok nem lehetnek összetettek
\end{itemize}

\noindent \textbf{Kulcsok:}
\begin{itemize}
    \item segítségükkel megkülönböztetjük a sorokat (nem lehetnek ugyanolyanok)
    \item szuperkulcs: sorokat megkülönböztető oszlophalmaz
    \item kulcs: minimális elemszámú szuperkulcs
    \item elsődleges kulcs: a megkülönböztetésre választott kulcs
    \begin{itemize}
        \item egyedi érték kell legyen
        \item nem lehet benne NULL
    \end{itemize}
    \item külső kulcs: 1:N és M:N kapcsolatok leírására
    \begin{itemize}
        \item másik táblázat elsődleges kulcsa által felvett érték, vagy NULL lehet
    \end{itemize}
\end{itemize}

\noindent \textbf{Lekérdezések:}
\begin{itemize}
    \item \texttt{SELECT [ALL  | DISTINCT | TOP n [PERCENT] ] \{ <kifejezés lista> | * \}}
    \item \texttt{FROM \{ <táblázat név> > \} [másodnév] [, ...]}
    \item \texttt{[WHERE <kiválasztási feltétel>]}
    \item \texttt{[GROUP BY <oszlopnév lista>]}
    \item \texttt{[HAVING <kiválasztási feltétel>]}
    \item \texttt{[ORDER BY \{<oszlop név> | <egész áll>\} [ASC|DESC] [, ...] ]}
    \item pl. kik kaptak elégtelent?
    \item \texttt{SELECT neptun FROM vizsga WHERE jegy=1;}
    \item Ahhoz, hogy a nevet is megtudjuk, kell a másik tábla is
    \item \texttt{SELECT hallgato.nev, vizsga.jegy FROM hallgato, vizsga, WHERE hallgato.neptun = vizsga.neptun and jegy=1;}
\end{itemize}