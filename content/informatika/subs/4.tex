\subsection{Ismertesse a fontosabb adatszerkezeteket (tömb, kapcsolt lista, verem, sor, fa, gráf) és a pointer fogalmát!}

\noindent \textbf{Pointer:}
\begin{itemize}
    \item nem adatot, hanem címet tárol
    \item az adatot indirekt módon tudjuk elérni
    \item típusa: ilyen típusú adatot tartalmaz a cím, amire a pointer mutat
\end{itemize}

\noindent Adatokat különböző adattárolókban lehet tárolni, melyek más-más módon működnek, attól függően érdemes kiválasztani, melyiket szeretnénk használni, hogy mire használjuk őket.

\noindent \textbf{Tömb:}
\begin{itemize}
    \item azonos típusú adatok
    \item az adatok indexelve vannak (számozva)
    \item az index segítségével férhetünk hozzá bármelyik adathoz (\textbf{kicímzés} lehetséges!!!!)
    \item egymást követő memóriacímeken van tárolva
    \item deklarálásnál meg van szabva, hány elemű
    \item egy adott elemhez konstans idővel, bejárás nélkül férünk hozzá
    \item beszúrás/törlés nem hatékony
    \item vannak lineárisak és többdimenziósak is: mátrixok
    \item rendezetlen tömbben lineáris keresés: $O(n)$
    \item rendezett tömbben bináris keresés (intervallumfelezés): $O(\log_2 n)$
\end{itemize}

\begin{figure}[H]
    \centering
    \includegraphics[width=0.4\textwidth]{res/imgs/info_4_array.png}
\end{figure}

\noindent \textbf{Kapcsolt lista:}
\begin{itemize}
    \item Adatok olyan sorrendje, ahol a sorrendet a pointerek határozzák meg
    \item beszúrás/törlés hatékony
    \item bejárás, valamint \textbf{egy elem keresése} is lineáris bonyolultságú
    \item lehet egy,- és kétirányú lánc
    \item végjel: NULL pointer
    \item az adatok sorrendje a memóriában tetszőleges, csak a pointereket változtatjuk
\end{itemize}

\begin{figure}[H]
    \centering
    \includegraphics[width=0.8\textwidth]{res/imgs/info_4_linked_list.png}
\end{figure}

\newpage

\noindent \textbf{Gráf:}
\begin{itemize}
    \item sorszámozott csomópontok és az azokat összekötő élek halmaza
    \item szomszédok: összekötött csomópontok
    \item deg(u): egy adott csomópont foka a befutó élek száma
    \item ha deg(u) = 0, akkor a csomópont izolált pont
    \item egyik csomópontból ($v_0$) egy másikba ($v_n$) haladó élek halmaza az út
    \item P($v_0, \dots, v_n$) út zárt, ha $v_0 = v_n$
    \item egy út egyszerű, ha minden pontja különböző
    \item kör: 3-nál hosszabb egyszerű, zárt út
    \item összefüggő: olyan gráf, melynek bármelyik 2 pontja közt létezik út
    \item teljes: minden csomópont mindegyik másikkal össze van kötve
    \item címkézett: az élekhez súlyokat rendelünk
    \item súlyozott: nemnegatív címkék
    \item irányított: az éleknek iránya van
    \item tárolásuk: szomszédsági mátrix: $a_{ij} = 1$, ha i-ből j-be halad él, egyébként 0
\end{itemize}

\begin{figure}[H]
    \centering
    \begin{minipage}[b]{0.3\textwidth}
        \centering
        \includegraphics[width=\textwidth]{res/imgs/info_4_graph_simple.png}
        \caption*{Egyszerű gráf}
    \end{minipage}
    \hfill
    \begin{minipage}[b]{0.3\textwidth}
        \centering
        \includegraphics[width=\textwidth]{res/imgs/info_4_graph_complete.png}
        \caption*{Teljes gráf}
    \end{minipage}
    \hfill
    \begin{minipage}[b]{0.3\textwidth}
        \centering
        \includegraphics[width=\textwidth]{res/imgs/info_4_graph_weighted.png}
        \caption*{Súlyozott gráf}
    \end{minipage}
\end{figure}

\noindent \textbf{Fa:}
\begin{itemize}
    \item Köröket nem tartalmazó gráf
    \item elemek véges (T) halmaza, mely
    \begin{itemize}
        \item tartalmaz egy kitüntetett gyökérelemet
        \item a többi elem nemnulla diszjunkt részfája T-nek
    \end{itemize}
    \item tárolás:
    \begin{itemize}
        \item info(k) – az elem adatai
        \item gyermek – az első gyermek indexe
        \item testvér – az első testvér indexe
    \end{itemize}
    \item bináris fa: egy szülőnek max 2 leszármazottja lehet
\end{itemize}

% [Diagram: fa szerkezet és tömbös tárolás]

\noindent \textbf{Veremtár/stack:}
\begin{itemize}
    \item LIFO: Last In First Out
    \item Push: elemet a verembe rak
    \item Pop: elemet leemel a verem tetejéről
    \item alkalmazások: függvényhívás, rekurzió, böngésző ,,vissza”, szövegszerkesztő ,,undo” gombja
    \item SP: stack pointer: a verem tetejét mutatja
\end{itemize}

\noindent \textbf{Sor/Queue:}
\begin{itemize}
    \item FIFO: First In First Out
    \item 2 mutató: WP, RP (write, read)
    \item írás/olvasás előtt/után a mutatót növelni kell
\end{itemize}

% [Diagram: sor puffer RP/WP mutatókkal]