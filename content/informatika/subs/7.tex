\subsection{Az operációs rendszer céljai, feladatai. Folyamatok kommunikációja. Ütemezési algoritmusok az operációs rendszerben. Termelő-fogyasztó probléma. Postaláda kezelés. Szemaforok.}

\noindent \textbf{Operációs rendszer:}
\begin{itemize}
    \item egy program (rendszer), amely közvetítő szerepet játszik a számítógép felhasználója és a számítógép hardvere között
    \item felhasználói programok végrehajtása, felhasználói feladatmegoldás könnyítése
    \item a számítógép rendszer használatának kényelmesebbé tétele
    \item a számítógép hardver kihasználásának hatékonyabbá tétele
    \item koordinálja és vezérli a hardver erőforrások különböző felhasználók különböző programjai által történő használatát
    \item \textbf{alkalmazói-programok:} hogyan legyenek felhasználva a rendszer erőforrások a felhasználók számítási problémáinak megoldásához
    \item felhasználók: emberek, gépek, más számítógépek
    \item részei:
    \begin{itemize}
        \item Kernel
        \item Shell
        \item Utility-k
    \end{itemize}
\end{itemize}

\noindent \textbf{Kernel:}
\begin{itemize}
    \item memória rezidens főprogram
    \item folyamat kezelés
    \item memória kezelés
    \item háttértár kezelés
    \item I/O kezelés
    \item fájl kezelés
    \item védelmi rendszer
    \item hálózat elérés támogatása
\end{itemize}

\noindent \textbf{Folyamat kezelés:}
\begin{itemize}
    \item folyamat: egy végrehajtás alatt lévő program
    \item erőforrásokra van szüksége
    \item operációs rendszer feladata:
    \begin{itemize}
        \item folyamat létrehozása, törlése
        \item folyamat felfüggesztése, újraindítása (multitaszk)
        \item eszközök biztosítása a folyamatok szinkronizációjához, kommunikációjához
    \end{itemize}
    \item folyamat általában műveletek végrehajtása meghatározott sorrendben
    \item folyamat elkezdődik és befejeződik
    \item a következő részművelet végrehajtása akkor kezdődhet, ha az előző befejeződött
    \item op. rsz. több folyamatból áll
    \item hatékonyabb erőforrás kihasználás
    \item feladat végrehajtás gyorsítása
    \item többféle feladat egyidejű végrehajtása
\end{itemize}

\noindent \textbf{Folyamatok típusai:}
\begin{itemize}
    \item független: egymás működését nem befolyásolják
    \item versengő: nem ismerik egymást, de közös erőforrásokon kell osztozzanak
    \item együttműködő: ismerik egymást, együtt dolgoznak egy feladat megoldásán, információt cserélnek
\end{itemize}

\noindent Több egymással párhuzamosan futó folyamat gyakran kommunikál közösen használt memóriaterületek segítségével. Ezek a területek nem érhetők el egyidejűleg a folyamatok számára. Az egyidejű hozzáférés kizárása \textbf{szemaforok} (bináris, nem bináris) segítségével történik.

\noindent \textbf{Termelő-fogyasztó probléma:}

\begin{figure}[H]
    \centering
    \includegraphics[width=0.8\textwidth]{res/imgs/info_7_prod_cons.png}
\end{figure}

\begin{itemize}
    \item a közös adatterületet egyszerre csak egy folyamat használhatja
    \item kölcsönös kizárás esete nem csak közös memória esetén lép fel (pl. nyomtató közös használata)
\end{itemize}

\noindent \textbf{Megoldás: vezérlés szemafor segítségével:}

\begin{figure}[H]
    \centering
    \includegraphics[width=0.8\textwidth]{res/imgs/info_7_semaphore_solution.png}
\end{figure}

\begin{itemize}
    \item mielőtt egy folyamat elkezdené használni az erőforrást, ellenőriznie kell, hogy az szabad-e
    \item csak akkor kezdheti használni, ha a szemafor szabadot jelzett, egyébként vár
\end{itemize}

\begin{figure}[H]
    \centering
    \includegraphics[width=1.0\textwidth]{res/imgs/info_7_semaphore_flow.png}
\end{figure}

\begin{itemize}
    \item kritikus szekció, kritikus szakasz, kritikus régió
    \item oszthatatlan művelet: \textcolor{red}{P primitív}, \textcolor{green}{V primitív}
\end{itemize}

\begin{figure}[H]
    \centering
    \includegraphics[width=1.0\textwidth]{res/imgs/info_7_pv_primitives.png}
\end{figure}

\begin{itemize}
    \item P primitív: foglaltra állítás
    \item V primitív: szabadra állítás
\end{itemize}

\noindent \textbf{Postaláda kezelés:}
\begin{itemize}
    \item postaláda: olyan közös adatterület, ahová egynél több üzenet írható
\end{itemize}

\begin{figure}[H]
    \centering
    \includegraphics[width=1.0\textwidth]{res/imgs/info_7_mailbox.png}
\end{figure}

\begin{itemize}
    \item újabb szemaforok a vezérléshez:
    \begin{itemize}
        \item Tele
        \item Üres
    \end{itemize}
    \item ezekben egész számot tárolunk
    \item 3 db szemafor a vezérléshez:
    \begin{itemize}
        \item S: kölcsönös kizárást megvalósító szemafor (0 = foglalt, 1 = szabad)
        \item Tele: tele helyek száma, nem bináris
        \item Üres: üres helyek száma, nem bináris
    \end{itemize}
    \item \textcolor{red}{P primitív:} a paraméterül kapott szemafor értékének eggyel csökkentése (,,foglalt”)
    \item \textcolor{green}{V primitív:} a paraméterül kapott szemafor értékének eggyel növelése (,,szabad”)
\end{itemize}

\begin{figure}[H]
    \centering
    \includegraphics[width=1.0\textwidth]{res/imgs/info_7_prod_cons_buffer.png}
\end{figure}

\noindent \textbf{Pszeudokód:}
\begin{multicols}{2}
\begin{itemize}
    \item[] \textbf{Termelő:}
    \item P(Üres);
    \item P(S);
    \item Írás a memóriába;
    \item V(S);
    \item V(Tele);
\end{itemize}
\columnbreak
\begin{itemize}
    \item[] \textbf{Fogyasztó:}
    \item P(Tele);
    \item P(S);
    \item Olvasás a memóriából;
    \item V(S);
    \item V(Üres);
\end{itemize}
\end{multicols}