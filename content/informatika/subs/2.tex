\subsection{Ismertesse a Neumann-elveket! Ismertesse a Boole-függvények megadási módjait, a logikai alapműveleteket és a De Morgan-azonosságokat! Ismertesse a kombinációs és szekvenciális hálózatok alapvető típusait (S-R, J-K, D tárolók)!}

\noindent \textbf{Neumann elvek (1945):}
\begin{itemize}
    \item teljesen elektronikus működés
    \item kettes számrendszer használata
    \item szekvenciális művelet végrehajtás
    \item adatok és programok a belső memóriában
    \item univerzális felhasználás: Turing-gép
    \item öt funkcionális egység:
    \begin{itemize}
        \item aritmetikai egység
        \item központi vezérlőegység
        \item memóriák
        \item bemeneti-,
        \item és kimeneti egységek
    \end{itemize}
\end{itemize}

\noindent \textbf{Boole függvények:}
\begin{itemize}
    \item olyan matematikai függvények, melyek vagy 1, vagy 0 értéket vesznek fel
    \item bemenetei és kimenetei is logikai változók
    \item két független bemenet (A és B) esetén:
\end{itemize}

\begin{table}[H]
    \centering
    \small
    \begin{tabular}{|l|l|c|c|c|c|c|c|c|c|c|c|c|c|c|c|c|c|}
    \hline
    \textbf{A} & \textbf{B} & \boldmath$f_0$ & \boldmath$f_1$ & \boldmath$f_2$ & \boldmath$f_3$ & \boldmath$f_4$ & \boldmath$f_5$ & \boldmath$f_6$ & \boldmath$f_7$ & \boldmath$f_8$ & \boldmath$f_9$ & \boldmath$f_{10}$ & \boldmath$f_{11}$ & \boldmath$f_{12}$ & \boldmath$f_{13}$ & \boldmath$f_{14}$ & \boldmath$f_{15}$ \\ \hline
    0 & 0 & 0 & 0 & 0 & 0 & 0 & 0 & 0 & 0 & 1 & 1 & 1 & 1 & 1 & 1 & 1 & 1 \\ \hline
    0 & 1 & 0 & 0 & 0 & 0 & 1 & 1 & 1 & 1 & 0 & 0 & 0 & 0 & 1 & 1 & 1 & 1 \\ \hline
    1 & 0 & 0 & 0 & 1 & 1 & 0 & 0 & 1 & 1 & 0 & 0 & 1 & 1 & 0 & 0 & 1 & 1 \\ \hline
    1 & 1 & 0 & 1 & 0 & 1 & 0 & 1 & 0 & 1 & 0 & 1 & 0 & 1 & 0 & 1 & 0 & 1 \\ \hline
    \end{tabular}
\end{table}

\begin{itemize}
    \item ismert logikai műveletek: negálás, és, vagy, kizáró vagy (ezekre van műveleti jel is C-ben)
    \item f1: „és”, f7: „vagy”, f6: „kizáró vagy”, f8: „nem vagy”, f14: „nem és”, f0: „azonosan 0”, f15: „azonosan 1”.
    \item $n$ bemenet esetén $n$ db változó van, mindegyiknek 2 értéke, szóval a $2^n$ bemeneti kombinációhoz 2 elemet rendelünk, így $2^{2^n}$ db függvény lesz
    \item \textbf{De Morgan azonosságok:}
    $$ \overline{(A \wedge B)} = \bar{A} \vee \bar{B} $$
    $$ \overline{(A \vee B)} = \bar{A} \wedge \bar{B} $$
\end{itemize}

\noindent \textbf{Logikai kapuk:}
\begin{itemize}
    \item logikai ,,építőkockák”, melyek alapműveleteket valósítanak meg
    \item összekapcsolásukkal jöhet létre az aritmetikai művelet
\end{itemize}

\begin{figure}[H]
    \centering
    \includegraphics[width=0.6\textwidth]{res/imgs/info_2_logic_gates.png}
\end{figure}

\noindent \textbf{Kombinációs logikai hálózat:}
\begin{itemize}
    \item kimenete(i) csak a bemenet(ek)től függenek
    \item minden kimenetet egy függvény ír le: $F_1(X_1, X_2, \dots, X_n)$
\end{itemize}

\begin{figure}[H]
    \centering
    \includegraphics[width=0.4\textwidth]{res/imgs/info_2_comb_logic.png}
\end{figure}

\noindent \textbf{Példa: 1 bites fél összeadó}
\begin{itemize}
    \item bemenetek: A és B
    \item kimenet: Y és C (carry)-átvitel
\end{itemize}

\begin{table}[H]
    \centering
    \begin{tabular}{|c|c|c|c|}
    \hline
    \textbf{A} & \textbf{B} & \textbf{Y} & \textbf{C} \\ \hline
    0 & 0 & 0 & 0 \\ \hline
    0 & 1 & 1 & 0 \\ \hline
    1 & 0 & 1 & 0 \\ \hline
    1 & 1 & 0 & 1 \\ \hline
    \end{tabular}
\end{table}

\begin{itemize}
    \item Függvények:
    \item Y = A xor B
    \item C = A and B
\end{itemize}

\noindent \textbf{Szekvenciális logikai hálózat:}
\begin{itemize}
    \item függ a \textbf{bemenetektől} és a \textbf{hálózat belső állapotától}
    \item aszinkron: nincs ütemező órajel
    \item szinkron: csak órajelnél vált állapotot (CPU)
\end{itemize}

\begin{figure}[H]
    \centering
    \includegraphics[width=0.4\textwidth]{res/imgs/info_2_seq_logic.png}
\end{figure}

\newpage

\noindent \textbf{S-R tároló:}
\begin{itemize}
    \item S: set, 1-re állítja a kimenetet
    \item R: reset, 0-ra
    \item S = 0 és R = 0, akkor tartja az értékét (memória)
    \item S = 1 és R = 1 érvénytelen, mert kimenet és a kimenet negáltja is 0
\end{itemize}

\begin{figure}[H]
    \centering
    \includegraphics[width=0.4\textwidth]{res/imgs/info_2_sr_latch.png}
\end{figure}

\begin{table}[H]
    \centering
    \begin{tabular}{|c|c|c|c|}
    \hline
    \textbf{S} & \textbf{R} & \textbf{Q} & \boldmath$\bar{Q}$ \\ \hline
    0 & 0 & latch & latch \\ \hline
    0 & 1 & 0 & 1 \\ \hline
    1 & 0 & 1 & 0 \\ \hline
    1 & 1 & 0 & 0 \\ \hline
    \end{tabular}
\end{table}

\noindent \textbf{J-K tároló:}
\begin{itemize}
    \item működése az S-R tárolóéhoz hasonló, viszont ez egy szinkron működésű szekvenciális logikai hálózat
    \item J: set, 1-re állít
    \item K: reset, 0-ra állít
    \item S = 0, R = 0, tartja az értékét (memória)
    \item S = 1, r = 1, megváltoztatja az értékét az előző kimenet negáltjára
\end{itemize}

\begin{figure}[H]
    \centering
    \includegraphics[width=0.4\textwidth]{res/imgs/info_2_jk_latch.png}
\end{figure}

\begin{table}[H]
    \centering
    \begin{tabular}{|c|c|c|c|c|}
    \hline
    \textbf{C} & \textbf{J} & \textbf{K} & \textbf{Q} & \boldmath$\bar{Q}$ \\ \hline
    $\uparrow$ & 0 & 0 & latch & latch \\ \hline
    $\uparrow$ & 0 & 1 & 0 & 1 \\ \hline
    $\uparrow$ & 1 & 0 & 1 & 0 \\ \hline
    $\uparrow$ & 1 & 1 & toggle & toggle \\ \hline
    x & 0 & 0 & latch & latch \\ \hline
    x & 0 & 1 & latch & latch \\ \hline
    x & 1 & 0 & latch & latch \\ \hline
    x & 1 & 1 & latch & latch \\ \hline
    \end{tabular}
\end{table}

\noindent \textbf{D tároló:}
\begin{itemize}
    \item Az S-R tárolót kiegészítjük „és” kapukkal, valamint egy léptető órajellel: Clk
    \item Az egyetlen bemenetet kettéágaztatjuk, egyik felét negáljuk $\to$ nem lehet S és R egyszerre 1
    \item ,,statikus RAM”
    \item gyors elérés
    \item bonyolult felépítés
    \item CPU regiszterei, cache
\end{itemize}

% [Diagram: D tároló kapuszintű megvalósítás]