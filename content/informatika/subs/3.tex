\subsection{A számítógép felépítése. Memóriák. CPU részei. Utasítás ciklus. Szubrutinhívás. Interrupt. Közvetlen memória hozzáférés.}

\noindent \textbf{Hardver:}
\begin{itemize}
    \item elektronikus/mechanikus alkatrészek összessége
    \item kézbe vehető
    \item szoftver nélkül nem üzemképes
\end{itemize}

\noindent \textbf{Szoftver:}
\begin{itemize}
    \item számítógépet működtető programok
\end{itemize}

\noindent \textbf{Firmware:}
\begin{itemize}
    \item a kettő együtt
    \item olyan szoftver, mely egy hardverben található
    \item pl PC-nél bekapcsoláskor elinduló ROM-BIOS az alaplap egyik integrált áramkörébe töltött program
\end{itemize}

\noindent \textbf{Logikai felépítés és funkciók:}
\begin{itemize}
    \item \textbf{Busz:} adatok, címek, vezérlő jelek továbbítása
    \item \textbf{CPU:} műveletvégzés
    \item \textbf{Memóriák:} adattárolás
    \item \textbf{Perifériák:} kapcsolat a külvilággal
\end{itemize}

\begin{figure}[H]
    \centering
    \includegraphics[width=0.8\textwidth]{res/imgs/info_3_bus.png}
\end{figure}

\begin{itemize}
    \item busz: párhuzamos (manapság soros) jelköteg
    \item busz szélességek: egyszerre átvihető adat mérete
    \item adatbusz: 8, 16, 32, 64 bit, szinte mindig akkumulátor regiszter mérete
    \item címbusz: memória max méretét adja meg
\end{itemize}
\newpage
\noindent \textbf{Memóriák – tárolók hierarchia szintjei:}
\begin{itemize}
    \item \textbf{regiszterek:}
    \begin{itemize}
        \item a CPU belső tárolói
        \item nagyon gyors hozzáférés az adatokhoz
        \item közvetlenül a CPU-ban találhatóak
        \item utasítások végrehajtása
        \item adatok köztes tárolása
    \end{itemize}
    \item \textbf{cache:}
    \begin{itemize}
        \item kis méretű, gyors, közel a processzorhoz
        \item az aktuálisan leggyakrabban használt adatokat, utasításokat tárolja
        \item cél: memória-hozzáférési idő csökkentése, rendszer teljesítményének javítása
    \end{itemize}
    \item \textbf{operatív tár} (RAM – random access memory):
    \begin{itemize}
        \item adatok, utasítások átmeneti tárolása
        \item nagyobb kapacitás, mint a cache és regiszterek, de lassabb hozzáférés
        \item a futó programok számára ideiglenesen tárolódnak itt adatok
    \end{itemize}
    \item \textbf{Mágneses / SSD direkt elérésű háttértár:}
    \begin{itemize}
        \item mágneses lemezeket (merevlemezeket) vagy szilárdtest meghajtókat (SSD) használ
        \item nagyobb kapacitás
        \item adatok hosszútávú tárolása
    \end{itemize}
    \item \textbf{Szekvenciális elérésű háttértár}
    \begin{itemize}
        \item szalag típusú adathordozók
        \item adatok egymás után találhatóak
        \item ilyen sorrendben történik a hozzáférés
        \item lassabb
    \end{itemize}
\end{itemize}

\begin{figure}[H]
    \centering
    \includegraphics[width=0.6\textwidth]{res/imgs/info_3_memory_pyramid.png}
\end{figure}
\newpage
\noindent \textbf{CPU:}
\begin{itemize}
    \item központi feldolgozó egység
    \item operatív memóriából olvassa be a program utasításait és az adatokat
    \item az utasításokat dekódolja, végrehajtja
    \item eredmények operatív memóriában
    \item buszok vezérlése: címek kiküldése, vezérlőjelek, memória/periféria címzése, adat bekérése/kiküldése buszon (olvasás/írás)
    \item órajel működteti $\to$ szekvenciális
    \item gyorsítani lehet: cache, utasítás végrehajtó egység számának növelése
\end{itemize}

\noindent \textbf{CPU részei:}
\begin{itemize}
    \item regiszterek:
    \begin{itemize}
        \item processzoron belüli memória
        \item adatok, címek, műveletek eredményei
        \item az utasítások a busz nélkül elérik
    \end{itemize}
    \item ALU
    \begin{itemize}
        \item aritmetikai-logikai egység
        \item aritmetikai műveletek: összeadás, kivonás, a++, a--, 2-es komplemens képzés, néha egész számok szorzása/osztása, bitek eltolása
        \begin{itemize}
            \item lebegőpontos műveletekhez szoftver, vagy társ(-, vagy co)processzor
        \end{itemize}
        \item logikai műveletek: negálás, és, vagy, kizáró vagy, összehasonlítás: kivonás, eredménye csak flag-ekben jelenik meg
    \end{itemize}
\end{itemize}

\noindent \textbf{Utasítás ciklus:}
\begin{itemize}
    \item végrehajtandó utasítás címe programszámláló regiszterben
    \item utasítás beolvasása a memóriából (fetch)
    \item utasítás dekódolása, ha vannak, akkor paraméterek beolvasása memóriából
    \item utasítás végrehajtása, eredmény tárolása (ha van)
    \item következő utasítás címének megadása: programszámláló regiszter a végrehajtott utasítás hosszával inkrementálódik
\end{itemize}

\noindent \textbf{Szubrutinhívás:}
\begin{itemize}
    \item egy korábbi kódrészlet újrahívása
    \item a következő utasítás címe veremtárba kerül, a program oda fog visszatérni
    \item újra használható a kód
\end{itemize}

\noindent \textbf{Interrupt:}
\begin{itemize}
    \item megszakítja az adott utasítást
    \item eltárolja, hogy hova kell visszatérjen
    \item megszakítás kezelése (ISR – interrupt service routine)
    \item visszatérés oda, ahol félbe lett szakítva
\end{itemize}

\noindent \textbf{Közvetlen memória hozzáférés:}
\begin{itemize}
    \item DMA – direct memory access
    \item a művelet közben nincs nagy számítási igény: buszt kell vezérelni, címet növelni 1-gyel
    \item speciális áramkör: DMA vezérlő
    \item a CPU megmondja a DMA vezérlőnek, hogy honnan, hova, hány bájtot
    \item a DMA vezérlő elkéri a buszt a CPU-tól, ha annak épp nincs szüksége rá
    \item DMA vezérlő átvisz 1 bájtot a perifériából a saját adatregiszterébe
    \item a DMA vezérlő által elkért buszon átvisz 1 bájtot a memóriába (írás esetén fordítva)
    \item DMA vezérlő visszaadja a buszt
    \item DMA újra elkéri a buszt
    \item az utolsó bájt átvitele után nem kéri el a buszt, IRQ (interrupt request) -val jelzi, hogy az átvitel kész
    \item busz arbitáció: busz elkérése
\end{itemize}

\begin{figure}[H]
    \centering
    \includegraphics[width=0.8\textwidth]{res/imgs/info_3_dma.png}
\end{figure}