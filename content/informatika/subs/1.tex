\subsection{Mutassa be a Turing-gép felépítését és működését! Definiálja az eljárás (algoritmus) fogalmát!}

\noindent \textbf{Elméleti számítástudomány:} matematikai \\
\textbf{Számítástechnika:} elméleti és gyakorlati megvalósítás, technológia

\noindent \textbf{Turing gép:} elméleti „számítógép”. Részei:
\begin{itemize}
    \item \textbf{végtelen, cellákra osztott szalag}
    \begin{itemize}
        \item egy cellában lehet szimbólum, vagy lehet üres
        \item az adatok, műveletek, illetve eredmények cellái véges számúak, ezeken túl a szalag üres
    \end{itemize}
    \item \textbf{Író/olvasó fej:}
    \begin{itemize}
        \item egyszerre egy cellával foglalkozik
        \item írhat, olvashat, törölhet
        \item a szalagon jobbra/balra lépkedhet (változtatás nélkül)
    \end{itemize}
    \item \textbf{Vezérlőegység:}
    \begin{itemize}
        \item állapotai számozva
        \item véges állapot (véges állapotú automata)
        \item helyettesítési táblázat adja meg a működést
        \begin{itemize}
            \item állapot + művelet + adat $\to$ új állapot + eredmény + fejmozgás
        \end{itemize}
    \end{itemize}
\end{itemize}

\noindent \textbf{Turing gép:}
\begin{itemize}
    \item matematikailag 5-10 elemből álló szabályhalmaz
    \item informatikailag:
    \begin{itemize}
        \item szalag = memória
        \item vezérlőegység = CPU
        \item fej = busz
    \end{itemize}
    \item alkalmas rekurzióra $\to$ veremtár (stack)
\end{itemize}

\noindent \textbf{Eljárások, algoritmusok:}
\begin{itemize}
    \item emberi nyelven megfogalmazott feladat, cselekménysorozat
    \item megoldási eljárás (,,algoritmus jelölt”), program írható
    \item egy megoldási eljárás akkor \textbf{algoritmus}, ha \textbf{bármilyen} bemenet esetén \textbf{véges számú lépés} után eredményt kapunk (A Turing-gép megáll)
    \item az algoritmusra nem létezik formális matematikai definíció
\end{itemize}