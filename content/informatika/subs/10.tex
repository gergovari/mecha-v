\subsection{Hálózati kommunikáció, OSI/ISO modell. Hálózati elsőbbségi elvek. Az interneten használt kommunikációs protokollok. IP cím, maszkolás, DNS rendszer.}

\noindent \textbf{Hálózatok:}
\begin{itemize}
    \item számítógépeket gyakran kötjük adathálózatba
    \item erőforrás megosztás
    \item csoportmunka
    \item kommunikációs platform
    \item történet: a meglévő telefonhálózatot használták adatátvitel céljára
    \item ma adatátvitelre használt hálózaton telefonálunk, nézünk ,,TV"-t
\end{itemize}

\noindent \textbf{OSI/ISO (open systems interconnection) modell:}
\begin{enumerate}
    \item fizikai: összeköttetés, áramkörök
    \item adatkapcsolati: hibajavítás, forgalomvezérlés
    \item hálózat: ismétlés, darabolás, blokk, hálózati cím
    \item szállítási: topológiamentes adatszállítás érpár közt
    \item viszonylati (együttműködési): párbeszéd, kommunikáció fel-, és újraépítése
    \item megjelenítési (ábrázolási): szintakszis alkalmazások irányában
    \item alkalmazási: felhasználó programja (pl. böngésző, facebook app)
\end{enumerate}

\noindent \textbf{Hálózati cím:} egyértelműen meghatározza, kitől jön és kinek megy az adat. Hálózaton belül egyedi.

\noindent \textbf{Hálózat típusok:}
\begin{itemize}
    \item PAN: personal area network
    \item LAN: local area network
    \item WAN: wide area network
    \item topológiák: busz, csillag, gyűrű
\end{itemize}

\noindent \textbf{(Ethernet) hálózat elemei:}
\begin{itemize}
    \item repeater/hub: jelerősítő (csillag közepén)
    \item bridge/switch: cím/port párosítások megtanulása, forgalom irányítása
    \item router: útvonalválasztó – címváltoztatás is történik
\end{itemize}

\noindent \textbf{Hálózati elsőbbségi elvek:}
\begin{itemize}
    \item egy hálózaton több node is található, el kell dönteni, hogy ki kapcsolhat adásra – közlekedési analógia
    \item[1.] minden node kaphat időszeletet, ha szabad a csatorna, foglalt csatornánál véletlenszerű ideig vár, majd újravizsgál: kis terhelésnél hatékony
    \begin{itemize}
        \item[a.] ethernet ezt használja
    \end{itemize}
    \item[2.] fix időszelet áll rendelkezésre az adáshoz
    \begin{itemize}
        \item[a.] az időszelet lehet azonos az összes node-ra (token-ring)
        \item[b.] lehet változtatható: sok kis cella közlekedik, néhány a node-é
        \item[c.] mindig megvan a sávszélesség: nemzetközi vonalakhoz
    \end{itemize}
\end{itemize}

\noindent \textbf{Logikai elrendezések:}
\begin{itemize}
    \item kliens/szerver modell: központi erőforrású informatikai rendszer
    \item peer-to-peer modell: minden node rendelkezhet megosztott erőforrással (pl filecserélő hálózat)
\end{itemize}

\noindent \textbf{Az interneten használt kommunikációs protokollok:}
\begin{itemize}
    \item RFC: request for comments
    \begin{itemize}
        \item IETF és más szervezetek adják ki
        \item dokumentumtípus, protokollokat, szabványokat írnak le benne
        \item HTTP: Hypertext Transfer Protocol
    \end{itemize}
\end{itemize}

\noindent \textbf{IP cím (RFC -791):}
\begin{itemize}
    \item minden internetes protokollt használó host-nak egy, 32 bit-es (= 4 byte (v4)) címe van
    \item a címet decimálisan ábrázoljuk, pontokkal elválasztva, pl. 152.66.25.4
    \item hálózati maszk: direktben elérhető gépek számításához
    \item ARP protokoll használata: IP$\leftrightarrow$MAC address (LAN szegmensen belül)
    \item Gateway/átjáró: annak a node-nak a címe, amelyik a maszkon kívüli címek eléréséhez kell (router)
    \item legyen gépünk ip-je 152.66.25.4, maszk: 255.255.255.0, átjáró: 152.66.25.254
    \item a forrás és célcímet a maszkkal bináris ÉS művelettel számítjuk
\end{itemize}

\noindent Küldjünk csomagot a 152.66.25.13 célcímre.
\begin{verbatim}
152. 66. 25. 4 &        152. 66. 25. 13 &
255.255.255.0           255.255.255.0
----------------        ----------------
152.66.25.0             152.66.25.0 ugyanaz jött ki (== igaz)
\end{verbatim}
A másik gép közvetlenül (etherneten) elérhető

\noindent Kérdezzünk le a 217.20.130.97-címről:
\begin{verbatim}
152. 66. 25. 4 &        217. 20.130. 97 &
255.255.255.0           255.255.255.0
----------------        ----------------
152.66.25.0             217. 20.130.0 nem ugyanaz (==hamis).
\end{verbatim}
A kérést az átjárónak címezzük (152.66.25.254) (etherneten), majd ő továbbítja

\begin{itemize}
    \item bővítés (címtartomány betelik): ip v6: 8 db 16 bites szám hexadecimálisan, kettőspontok között, 0-k elhagyhatóak
    \item első 4 szám: hálózat, második 4 szám: ezen belüli gép
    \item pl. facebook IPv6 címe: 2a03:2880:f10c:83:face:b00c:0:25de
\end{itemize}

\noindent \textbf{DNS rendszer:}
\begin{itemize}
    \item domain name system
    \item az emberek számára nehezen megjegyezhetőek az IP címek
    \item a DNS név alapján találja meg a keresett szerver IP címét
    \item a gépeket névvel látják el, ezekhez kapcsolódnak az IP címek
    \item hierarchikus adatszerkezet alakul ki
    \item a felhasználó beírja egy cím nevét
    \item ha a cache nem ismeri a címet, akkor le kell kérdezi a helyi DNS kiszolgálót
    \item ez megkérdezi a gyökér DNS-szervert, az válaszol, majd a helyi DNS kérdez le újra az válaszban adott szervertől
    \item megkapja a keresett IP címet
\end{itemize}

\noindent \textbf{Keresés: mogi.bme.hu oldal ip címe ?}
\begin{itemize}
    \item TLD szerver: hu:$\rightarrow$ ns.nic.hu
    \item ns.nic.hu: bme.hu $\rightarrow$ ns.bme.hu
    \item ns.bme.hu: mogi.bme.hu$\rightarrow$delta.inflab.bme.hu
    \item delta.inflab.bme.hu$\rightarrow$centos.mogi.bme.hu
    \item (ip: 152.66.24.184)
\end{itemize}

\begin{itemize}
    \item reverse lookup: ip címből tud nevet mondani
\end{itemize}

\begin{figure}[H]
    \centering
    \includegraphics[width=1.0\textwidth]{res/imgs/info_10_dns.png}
\end{figure}