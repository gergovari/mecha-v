\subsection{Információelmélet. Shannon hírközlés modellje. Forráskódolás. Prefix kód. Huffman-kód.}

\noindent \textbf{Shannon hírközlés modellje:}
\begin{itemize}
    \item hírközlés során egy üzenetet juttatunk el térben és/vagy időben egy másik pontra
\end{itemize}

\begin{figure}[H]
    \centering
    \includegraphics[width=1.0\textwidth]{res/imgs/info_9_shannon_model.png}
\end{figure}

\begin{itemize}
    \item \textbf{Forrás/adó:}
    \begin{itemize}
        \item Tényleges folytonos, analóg forrás $\xrightarrow{a}$ Mintavételezés, kvantálás, forráskódolás $\xrightarrow{b}$
        \item Diszkrét értékekké alakítjuk időben és dimenzióban is, majd tömörítjük
    \end{itemize}
    \item \textbf{kódoló:}
\end{itemize}

\begin{figure}[H]
    \centering
    \includegraphics[width=0.8\textwidth]{res/imgs/info_9_channel_coding.png}
\end{figure}

\begin{itemize}
    \item \textbf{Csatornakódolás (hibajavító kódolás):} lehetővé teszi a zajos csatornán át a biztonságos(abb) üzenetátvitelt, a hibák jelzését és kijavítását.
    \item \textbf{csatorna:}
\end{itemize}

\begin{figure}[H]
    \centering
    \includegraphics[width=1.0\textwidth]{res/imgs/info_9_channel.png}
\end{figure}

\begin{itemize}
    \item modulátor $\rightarrow$ csatorna (Zaj, jeltorzulás) $\rightarrow$ demodulátor
    \item Átalakítja a kódolt üzenetet a csatornán átvihető jellé
    \item Eldönti, hogy a lehetséges leadható jelalakok közül melyiket adhatták le.
    \item \textbf{dekódoló:}
\end{itemize}

\begin{figure}[H]
    \centering
    \includegraphics[width=0.4\textwidth]{res/imgs/info_9_decoder.png}
\end{figure}

\begin{itemize}
    \item Kijavítja és/vagy jelzi a vett jelek hibáit, elvégzi a csatorna dekódolást.
    \item \textbf{vevő/nyelő:}
\end{itemize}

\begin{figure}[H]
    \centering
    \includegraphics[width=0.6\textwidth]{res/imgs/info_9_receiver.png}
\end{figure}

\begin{itemize}
    \item Forráskódolás inverze $\xrightarrow{a'}$ vevő
    \item A helyreállított üzenetet kitömöríti
    \item Értelmezi az üzenetet
\end{itemize}

\noindent \textbf{Információelmélet:}
\begin{itemize}
    \item információ: valamely véges számú, előre ismert esemény közül annak a megnevezése, hogy melyik következett be
    \item az információ mértéke azonos azzal a bizonytalansággal, amit megszüntet
    \item \textbf{Hartley:} $m$ számú, azonos valószínűségű esemény közül egy megnevezésével nyert információ: $I = \log_2 m$
    \item vagyis $\log_2 m$ db eldöntendő kérdéssel azonosítható egy elem
    \item \textbf{Shannon:} minél váratlanabb egy esemény, bekövetkezése annál több információt jelent
    \item Legyen $A = \{A_1, A_2, ..., A_m\}$ esemény-halmaz, az $A_1$ esemény valószínűsége $p_1$, az $A_m$ esemény valószínűsége $p_m$. Ekkor az $A_i$ esemény megnevezésével nyert információ: $I(A_i) = \log_2 \frac{1}{p_i} = -\log_2 p_i$
    \item Ha $p_i = 1/m$ az összes i-re, akkor visszakapjuk a Hartley féle definíciót.
\end{itemize}

\noindent \textbf{Forráskódolás:}
\begin{itemize}
    \item A forrás kimenetén véges sok elemből álló $A = \{A_1, A_2, ..., A_n\}$ halmaz elemei jelenhetnek meg. Az A halmaz elnevezése: forrásábécé.
    \item Üzenet: az A forrásábécé betűiből képzett, véges $A^{(1)}, A^{(2)}, ..., A^{(m)}$ sorozatok
    \item A lehetséges üzenetek halmaza: $\mathcal{A}$
    \item a kódolt üzenetek egy $B = \{B_1, B_2, ..., B_s\}$ szintén véges halmaz elemeiből épülnek fel, ahol B elnevezése: kódábécé
    \item kódszavak: B elemeiből képzett véges hosszúságú $B^{(1)}, B^{(2)}, ..., B^{(m)}$ sorozatok
    \item a lehetséges kódszavak halmaza: $\mathcal{B}$
    \item az f: $A \rightarrow \mathcal{B}$ és F: $\mathcal{A} \rightarrow \mathcal{B}$ függvényeket forráskódnak nevezzük
    \item az f leképezés a forrás minden egyes szimbólumához rendel egy kulcsszót
    \item egyértelműen dekódolható kód: egy f forráskód egyértelműen dekódolható, ha minden egyes B-beli sorozatot csak egy féle A-beli sorozatból állít elő
    \begin{itemize}
        \item feltétele: f és F is invertálható legyen
    \end{itemize}
    \item az állandó hosszúságú kódok egyértelműen dekódolhatóak, de nem mindig gazdaságosak: ASCII kód: 'A' $\leftrightarrow$ 65
\end{itemize}

\noindent \textbf{Prefix kód:}
\begin{itemize}
    \item a lehetséges kódszavak közül egyik sem folytatása a másiknak, bármely kódszó végéről levágva bármekkora szegmenst, nem kapunk másik kódszót
    \item a prefix kód egyértelműen dekódolható, de létezik nem prefix is, ami egyértelmű
    \item pl. $A = \{a,b,c,d\}$, $B = \{0,1\}$, $f(a) = 0$, $f(b) = 01$, $f(c) = 011$, $f(d) = 0111$, nem prefix, de egyértelműen dekódolható, mert 0-nál az új kód kezdődik
    \item legrövidebb átlagos szóhosszúságú prefix kód: \textbf{Huffman-kód}
\end{itemize}