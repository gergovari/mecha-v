\section{Informatika}

\subsection{A számítástudomány alapjai. Turing gép. Eljárások, algoritmusok.}

\subsection{A számítógép architektúrák alapjai. Boole függvények. Logikai kapuk. Kombinációs és szekvenciális logikai hálózatok. Tárolók: S-R, J-K, D.}

\subsection{A számítógép felépítése. Memóriák. CPU részei. Utasítás ciklus. Szubrutinhívás. Interrupt. Közvetlen memória hozzáférés.}

\subsection{Adatszerkezetek. Tömbök, kapcsolt listák, gráf, fa, verem, sor.}

\subsection{Algoritmusok. Bejárás, keresés, rendezés. Algoritmusok bonyolultsága. Rekurzió.}

\subsection{Az adatbázisok alapjai. Adatmodellezés. Kapcsolatok típusai. Relációs adatbázismodell. Relációk jellemzői. A relációs algebra műveletei. SQL alapok, lekérdezések.}

\subsection{Az operációs rendszer céljai, feladatai. Folyamatok kommunikációja. Ütemezési algoritmusok az operációs rendszerben. Termelő-fogyasztó probléma. Postaláda kezelés. Szemaforok.}

\subsection{Holtpont az operációs rendszerben. Holtpont kezelése. Holtpont észlelése. Holtpont megelőzés. Bankár algoritmus.}

\subsection{Shannon hírközlési modellje. Forráskódolás, prefix kód.}

\subsection{Hálózati kommunikáció, OSI/ISO modell. Hálózati elsőbbségi elvek. Az interneten használt kommunikációs protokollok. IP cím, maszkolás, DNS rendszer.}

\subsection{Az objektum fogalma, objektum-orientált elvek. Az osztály fogalma. Struktúrák. Tagfüggvények. Konstruktor. Destruktor. Statikus tagok. Barátság, friend függvények.}

\subsection{Operátorok túlterhelése az objektum orientált programozásban. C++ IO, new, delete operátorok túlterhelésének szabályai. Osztály hierarchiák.}

\subsection{Öröklődés, egységbe zárás az objektum-orientált programozásban. Protected osztálytagok. Kompozíció. Aggregáció. Többszörös öröklődés.}

\subsection{Polimorfizmus az objektum-orientált programozásban. Virtuális alaposztályok. Abstract osztály. Általánosított osztályok.}

\subsection{Standard Template Library a C++-ban. Tárolók. Bejárók. Algoritmusok. Függvényobjektumok.}

\subsection{Fuzzy halmazok alapjai, műveletek fuzzy halmazokon.}

\subsection{Fuzzy következtető módszer, defuzzifikációs módszerek.}

\subsection{Aggregációs operátorok, általános hatványközép, OWA.}

\subsection{A .net rendszer részei: GC, CIL, assembly-k. Esemény vezérelt programok felépítése windows alatt.}

\subsection{Grafikus adattárolás (vektor, raszter), alkatrész modellezési módszerek.}

\subsection{3D->2D vetítési algoritmusok, a window-viewport transzformáció.}

\subsection{Görbe közelítési módszerek: természetes spline, Bezier, Catmull-Rom görbék.}

\subsection{Láthatóság, árnyalás, megvilágítás, színmodellek, anyagmodellek.}

\subsection{Képfeldolgozás, konvolúció, élkeresés, szegmentálás, alakfelismerés.}

\subsection{Neurális hálózatok alapjai, a Perceptron, a Perceptron tanítása.}

\subsection{Felügyelt és felügyelet nélküli tanulás. Mesterséges neurális hálózatok.}

\subsection{Evolúciós algoritmusok, evolúció stratégiák, genetikus programozás.}

\subsection{Az "M" nyelv (Matlab) jellegzetességei: változók, vektorok és mátrixok, feltételes végrehajtás, ciklusok, számtani sorozatok, függvény definíció, diagram rajzolás.}

\subsection{Ismertesse az alábbi, mechatronikában tanult elvek programmal történő megvalósítását: állapotgép, ARMA modell.}